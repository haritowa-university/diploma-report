\sectioncentered*{Реферат}
\thispagestyle{empty}

% Зачем: чтобы можно было вывести общее число страниц.
% Добавляется единица, поскольку последняя страница -- ведомость.
\FPeval{\totalpages}{round(\getpagerefnumber{LastPage} + 1, 0)}

БГУИР ДП 1-39 03 01 019 ПЗ
\bigbreak
Харченко, А.К. Клиент-серверное программное средство обмена шифрованными сообщениями с iOS-клиентом : пояснительная записка к дипломному проекту / А.К. Харченко -- Минск : БГУИР, 2018. -- \totalpages~c.
\bigbreak
Пояснительная записка \totalpages~с., \totfig{}~рис., \tottab{}~табл., \toteq{}~формул, \totref{}~источников.
\bigbreak
\MakeUppercase{Программное средство, iOS-приложение, чат, сквозное шифрование}
\bigbreak

\textit{Цель проектирования}: разработка программного средства, предназначенного для обмена шифрованными сообщениями при помощи алгоритмов сквозного шифрования. 

\textit{Методология проведения работы}: в процессе решения поставленных задач использованы принципы и подходы экстремального программирования, методика разработки через тестирование.

\textit{Результаты работы} выделены основные аспекты и существующие проблемы процесса обмена шифрованными сообщениями. Кроме того, рассмотрены существующие средства. Выработаны функциональные и нефункциональные требования.

Была разработана архитектура программной системы, для каждой ее составной части было проведено разграничение реализуемых задач, проектирование, уточнение используемых технологий и собственно разработка. Были выбраны наиболее современные средства разработки, широко применяемые в индустрии. 

Полученные в ходе технико-экономического обоснования результаты о прибыли для разработчика, пользователя, уровень рентабельности, а также экономический эффект доказывают целесообразность разработки про\-екта.

\textit{Область применения результатов}: проект может использоваться частным сектором для безопасного обмена сообщениями и документами.