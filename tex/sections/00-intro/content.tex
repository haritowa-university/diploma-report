\sectioncentered*{Введение}
\addcontentsline{toc}{section}{Введение}

Люди давно поняли, что~коммуникация -- одно из~самых важных условий повышения как личной, так и~коммандной эффективности. Современный мир сложно представить без общения при~помощи всевозможных цифровых способов передачи информации: чатов, почты, \gls{voip}. Капитализация стартапа \textit{Slack} в~5 миллионов долларов\cite{slack:capitalization} доказывает актуальность мобильных чатов в~современном мире.

В 2018 году в~России разгорелся скандал с~одним из~крупнейших мессенджеров современности: \textit{Telegram}. Правительство России требует от~стартапа выдать ключи шифрования и~историю переписок пользователей, что~в~очередной раз поднимает вопросы безопасности общения и~передачи документов при~помощи сети интернет\cite{telegram:vs:rkn}.

Целью настоящего дипломного проектирования является разработка программного средства, которое устанавливается на~сервера заказчика и~предоставляет возможность обмена шифрованными сообщениями, используя алгоритм сквозного шифрования.

В пояснительной записке к~дипломному проекту излагаются детали поэтапной разработки приложения. В~первом разделе приведены результаты анализа литературных источников, рассмотрены особенности существующих систем-аналогов, выдвинуты требования к~проектируемому ПС, приведено описание функциональности проектируемого ПС, представлена спецификация функциональных требований. Во втором разделе приведены детали проектирования и~конструирования ПС. Результатом этапа конструирования является функционирующее программное средство. В~третьем разделе представлены способы верификации работоспособности ПС. В~четвертом разделе изложена инструкция по~эксплуатации ПС, а~в~пятом -- расчеты метрик производительности программного средства. Обоснования целесообразности создания программного средства с~технико-экономической точки зрения приведено в~шестом разделе. Итоги проектирования, конструирования программного средства, а~также соответствующие выводы приведены в~заключении.

Дипломный проект выполнен самостоятельно, проверен в~системе <<Атиплагиат>>. Процент оригинальности составляет \num{92.32} \%. Цитирования обозначены ссылками на~публикации, указанными в~«Списке использованных источников».
Скриншот приведен в~приложении.