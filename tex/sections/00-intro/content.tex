\sectioncentered*{Введение}
\addcontentsline{toc}{section}{Введение}

Люди давно поняли, что коммуникация -- одно из самых важных условий повышения как личной, так и коммандной эффективности. Современный мир сложно представить без общения при помощи всевозможных цифровых способов передачи информации: чатов, почты, \gls{voip}. Капитализация стартапа \textit{Slack} в 5 миллионов долларов \cite{slack:capitalization} доказываает актуальность чат-приложений в современном мире.

В 2018 году в России разгорелся скандал с одним из крупнейших мессенджеров современности: \textit{Telegram}. Правительство России требует от стартапа выдать ключи шифрования и историю переписок пользователя, что в очередной раз поднимает вопросы безопасности общения и передачи документов при помощи сети интернет. \cite{telegram:vs:rkn}

Помимо угрозы со стороны правительства стран, существует ещё вероятность ошибок в программных продуктах или наличия так называемых бекдоров. Отсутствие контроля над способом и локацией хранения данных, инфраструктурой, физическими серверами и способом реализации являются критическими недостатками для бизнеса большинства существующих программных решений, предоставляющих функицональность коммуникации с использованием сквозного шифрования.

Целью настоящего дипломного проектирования разработка программного средства, которое устанавливается на сервера заказчика и предоставляет возможность обмена шифрованными сообщениями, используя алгоритм сквозного шифрования.

В пояснительной записке к дипломному проекту излагаются детали поэтапной разработки приложения. В первом разделе приведены результаты анализа литературных источников, рассмотрены особенности существующих систем-аналогов, выдвинуты требования к проектируемому ПС, приведено описание функциональности проектируемого ПС, представлена спецификация функциональных требований. Во втором разделе приведены детали проектирования и конструирования ПС. Результатом этапа конструирования является функционирующее программное средство. В третьем разделе представлены способы верификации работоспособности ПС. В четвертом разделе изложена инструкция по эксплуатации ПС. Обоснования целесообразности создания программного средства с технико-экономической точки зрения приведено в пятом разделе. Итоги проектирования, конструирования программного средства, а также соответствующие выводы приведены в заключении.