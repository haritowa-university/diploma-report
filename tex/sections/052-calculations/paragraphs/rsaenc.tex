\subsubsection{} \textit{RSA} шифровка и~расшифровка.
\label{sec:eng:performance:rsaenc}

Метод \textit{rsaEncrypt} необходим для~зашифровки данных при~помощи алгоритма \textit{RSA}, а~метод \textit{rsaDecrypt} -- для~расшифровки данных, зашифрованных при~помощи \textit{RSA}. На~рисунке \ref{sec:eng:performance:rsaenc:image} представлены результаты производительности модуля \textit{RSA}.

\begin{figure}[h]
  \centering
    \includegraphics[width=0.75\textwidth]{inc/img/rsa_performance_test.jpg}
  \caption{Результаты тестов производительноси модуля RSA}
  \label{sec:eng:performance:rsaenc:image}
\end{figure}

\FPeval{\rsaEncMesaureMax}{0.00423}
\FPeval{\rsaEncMesaureMin}{0.00281}
\FPeval{\rsaEncMesaureAverage}{0.00352}
\FPeval{\perfRSAEnc}{clip(round(((\rsaEncryptMaxValue - (\rsaEncMesaureMax + \rsaEncMesaureMin + \rsaEncMesaureAverage) / 3) / \rsaEncryptMaxValue) * 100, 2))}

Рассчитаем отклонение от~предельно допустимого значения времени, затраченного на~шифрование при~помощи \textit{RSA}, подставив значения в~формулу (\ref{perfDifEquation}):
\begin{center}
\(\perfDev = (\num{\rsaEncryptMaxValue} - \frac{\num{\rsaEncMesaureMax} + \num{\rsaEncMesaureMin} + \num{\rsaEncMesaureAverage}}{\num{3}}) \cdot \frac{\num{1}}{\num{\rsaEncryptMaxValue}} \cdot 100 = \num{\perfRSAEnc} \, \text{\%}\)
\end{center}

\FPeval{\rsaDecMesaureMax}{0.236}
\FPeval{\rsaDecMesaureMin}{0.224}
\FPeval{\rsaDecMesaureAverage}{0.228}
\FPeval{\perfRSADec}{clip(round(((\rsaDecryptMaxValue - (\rsaDecMesaureMax + \rsaDecMesaureMin + \rsaDecMesaureAverage) / 3) / \rsaDecryptMaxValue) * 100, 2))}

Для дешифровки, при~помощи алгоритма \textit{RSA}, рассчитаем аналогичное значение:
\begin{center}
\(\perfDev = (\num{\rsaDecryptMaxValue} - \frac{\num{\rsaDecMesaureMax} + \num{\rsaDecMesaureMin} + \num{\rsaDecMesaureAverage}}{\num{3}}) \cdot \frac{\num{1}}{\num{\rsaDecryptMaxValue}} \cdot 100 = \num{\perfRSADec} \, \text{\%}\)
\end{center}