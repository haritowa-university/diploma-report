\section{Тестирование и~проверка работоспособности программного средства}
\label{sec:testing}

Тестирование программного обеспечения –- процесс анализа программного средства и~сопутствующей документации с~целью выявления дефектов и~повышения качества продукта\cite{kulikov_testing}.

Тестирование давно стало неотъемлимой частью процесса разработки, причём тестирование сопровождает практически весь цикл разработки. Существуют различные виды тестирования:
\begin{enumerate}
	\item \textit{Ручное тестирование} обычно производится сотрудниками, занимающими позицию тестировщика в~компании.
	\item \textit{Атвоматизированное тестирование} выполняется автоматически, а~код, описывающий процесс тестирования, пишется разработчиками \gls{pp} или~тестировщиками автоматизаторами.
\end{enumerate}