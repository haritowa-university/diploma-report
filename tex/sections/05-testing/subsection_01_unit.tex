\subsection{Модульное тестирование кодовой базы}
\label{sec:testing:unit}

В рамках данного дипломного проекта было принято решение сопровождать разрабатываемые классы модульными тестами. Ниже будет рассмотрено понятие модульного тестирования, методика разработки через тестирование и выбраны технологии, помогающие в разработке кода тестов.

\subsubsection{}
\label{sec:testing:unit:defenition}

Единичное тестирование, или~модульное тестирование -— процесс в~программировании, позволяющий проверить на~корректность единицы исходного кода, наборы из~одного или~более программных модулей вместе с~соответствующими управляющими данными, процедурами использования и~обработки.

Идея состоит в~том, чтобы писать тесты для~каждой нетривиальной функции или~метода. Это позволяет достаточно быстро проверить, не привело ли очередное изменение кода к~регрессии, то есть к~появлению ошибок в~уже оттестированных местах программы, а~также облегчает обнаружение и~устранение таких ошибок. 

Цель модульного тестирования —- изолировать отдельные части программы и~показать, что~по~отдельности эти части работоспособны\cite{wiki:unit}.

Кроме дополнительных гарантий корректности работы программы, модульное тестирование значительно повышает качество кода и~архитектуры самого приложения:

\begin{enumerate}
	\item Тесты заставляют отвественнее подходить к~проектированию зависимостей класса.
	\item Модульные тесты в~понятной любому человеку форме описывают задачи и~особенности работы класса или~метода.
	\item Тесты позволяют производить рефакторинг с~меньшей степенью регрессий.
	\item По классу тестов типа можно быстро оценить масштаб и~разнообразие задач, решаемых типом.
	\item Модульное тестирование вдохновляет писать код таким образом, что~позже можно легко заменять части программы на~альтернативную реализацию, сохраняя при~этом корректность работы остальных модулей.
\end{enumerate}

\subsubsection{}
\label{sec:testing:unit:mock}

При разработке тестов часто используются \mock \, и \stub \, объекты для подмены настоящих зависимостей типа на объекты, позволяющие:

\begin{itemize}
	\item вернуть подготовленные значения из методов зависимости;
	\item заменить настоящую реализацию на пустую, упрощая этап подготовки теста;
	\item эимитировать состояние, например \stub \, объект базы данных может быть сконигурирован так, будто это база данных с определёнными записями в ней;
	\item проверицифировать вызовы методов у объектов зависимостей.
\end{itemize}

\textit{Mock-объект} представляет собой конкретную фиктивную реализацию интерфейса, предназначенную исключительно для тестирования взаимодействия и относительно которого высказывается утверждение. \cite{wiki:mock}

В русской терминологии \stub \, объекты и функции иногда называются заглушками и определяются как объект или функция, не выполняющая никакого осмысленного действия, возвращающая пустой результат или входные данные в неизменном виде. \cite{wiki:stub}
\subsubsection{}
\label{sec:testing:unit:tdd}

В экстремальных методиках программирования существует \textit{методика разработки через тестирование}, которая подразумевает использование следующего алгоритма при разработке программного обеспечения:

\begin{enumerate}
	\item Написание интерфейса и протокола типа: описание протокола, который должен будет адоптировать разрабатываемый тип и написание необходимого и достаточного количества кода для успешной компиляции программы с новым типом, реализующим протокол.
	\item Написание тестов для типа. На этом этапе все тесты дожны выдавать ошибку, если тесты успешно проходят -- высоки шансы, что такие тесты ничего не тестируют.
	\item Написание необходимого количества кода для удовлетворения условий тестов. Результатом выполнения данного шага является успешное завершение всех разработанных ранее тестов.
	\item При обновлении требований или разработке большой декомпозированной задачи, пункты повторяются заново.
\end{enumerate}

Данный подход хорош тем, что позволяет установить требования для типа и продумать зависимости ещё до разработки, проверить работоспособность тестов при нерабочем коде и постоянно повышать покрытие кода тестами. При разработке текущего дипломного проекта активно применялся рассмотренный выше подход.

Как уже было отмечено, тестирование позволяет значительно повысить качество кода, пункт \ref{sec:testing:unit:mock} ввёл концепцию объектов пустышек, который демонстрирует гибкость и удобство правильно спланированных зависимостей, а пункт \ref{sec:testing:unit:tdd} определяет весь дальнейший процесс разработки.