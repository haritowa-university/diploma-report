\subsubsection{}
\label{sec:testing:unit:mock}

При разработке тестов часто используются \mock \, и \stub \, объекты для подмены настоящих зависимостей типа на объекты, позволяющие:

\begin{itemize}
	\item вернуть подготовленные значения из методов зависимости;
	\item заменить настоящую реализацию на пустую, упрощая этап подготовки теста;
	\item эимитировать состояние, например \stub \, объект базы данных может быть сконигурирован так, будто это база данных с определёнными записями в ней;
	\item проверицифировать вызовы методов у объектов зависимостей.
\end{itemize}

\textit{Mock-объект} представляет собой конкретную фиктивную реализацию интерфейса, предназначенную исключительно для тестирования взаимодействия и относительно которого высказывается утверждение. \cite{wiki:mock}

В русской терминологии \stub \, объекты и функции иногда называются заглушками и определяются как объект или функция, не выполняющая никакого осмысленного действия, возвращающая пустой результат или входные данные в неизменном виде. \cite{wiki:stub}