\subsubsection{}
\label{sec:testing:tech:xctest}

Ещё одной проблемой в мире тестирования iOS-приложений является отсутствие хороших стандартных инструментов. Apple предлоставляет тесовый фреймворк \textit{XCTest}, который позволяет писать любые виды тестов:

\begin{itemize}
	\item модульные;
	\item интеграционные;
	\item тесты пользовательского интерфейса;
	\item тесты производительности;
	\item и многие другие.
\end{itemize}

Несмотря на гибгость фреймворка, работа с ним сильно усложняется из-за подхода: набор тестов описывается наследником от \textit{XCTestCase}, подготовка перед тестами и сброс после тестов выполняется в переопределяемых методах, а каждый тест является отедльным методом. Такой подход заставляет вводить переменные в тестах, усложняет процесс фокусированной подготовки к группе тестов. Ещё одним минусом является набор методов для верификации тестов, который, в случае ошибки, не способен сгенерировать осмысленное описание. Длинна названия каждого метода становится очень большой, тестовый контекст разбросан по типу, чтение таких конфигураций не является тривиальной задачей, что портит идею использования модульных тестов в качестве приложения к документации.

\subsubsection{}
\label{sec:testing:tech:quick}
\textit{Quick} -- фреймворк для разработки тестов по методологии "движемой поведением разработки" для языков Swift и Objective-C, вдохновлённый RSpec, Specta и Ginkgo. \cite{github:quick}

Для решения описанных в пункте \ref{sec:testing:tech:xctest} проблем в проекте используется тестовый фреймворк \textit{Quick} и прилагающийся к нему набор методов верификации \textit{Nimble}.

\textit{Quick} рассматривает набор тестов как один метод, который формируется из трёх вспомогательных:

\begin{enumerate}
	\item \textit{it} содержит один или более вызовов верификации и отражает один тест;
	\item \textit{describe} -- служит для группировки остальных методов по тестируемому свойству или поведению;
	\item \textit{context} -- описывает некоторый контекст, в котором будут выполнятся вложенные тесты.
\end{enumerate}

К каждому из методов конфигурации можно добавить неограниченное количество методов подготовки и очистки, название тестов генерируется из иерархии используемых \textit{describe} и \textit{context}, а осмысленные сообщения об ошибках генерируют методы верификации из модуля \textit{Nimble}.

\subsubsection{}
\label{sec:testing:tech:cuckoo}
Для генерации \textit{Mock} и \textit{Stub} объектов используется библиотека \textit{Cuckoo}.

\textit{Cuckoo} состоит из двух частей: рантайм библиотеки и утилиты для коммандной строки, который генерирует моковый класс. Типичное использование библиотеки выглядит следующим образом:

\begin{enumerate}
	\item Добавление очередного класса в список файлов, для которого должны быть созданы \textit{Mock} и \textit{Stub}.
	\item Генерация нового файла со всем заглушками. Утилита для коммандной строки создаст наследника класса или пустой класс, реализующий требуемый протокол, который способен запоминать список вызванных методов.
	\item Создание \textit{Mock} объекта.
	\item Описание поведения для нового объекта. Обычно это сводится к инструкциям вида "если метод вызван с определённым параметром -- верни определённый результат".
	\item Передача мокового объекта в тестируемый класс.
	\item Вызов методов верификации на моковом объекте. Обычно это сводится к инструкциям вида "данный метод должен быть вызван с определёнными параметрами определённое количество раз".
\end{enumerate}

\subsubsection{}
\label{sec:testing:tech:result}
Таким образом, полный набор библиотек, использующихся для тестирования ПП, выглядит следующим образом:

\begin{itemize}
	\item \textit{RxTest};
	\item \textit{RxBlocking};
	\item \textit{XCTest} для тестов производительности;
	\item \textit{Quick};
	\item \textit{Nimble};
	\item \textit{Cuckoo};
	\item \textit{FoundationTesting} -- набор собственных расширений и утилит.
\end{itemize}