\subsubsection{}
\label{sec:testing:ci:jenkins}

\textbf{Jenkins} -- проект для~непрерывной интеграции с~открытым исходным кодом, написанный на~Java. Был ответвлён от~проекта Hudson, принадлежащего компании Oracle. Распространяется под~лицензией MIT. Позволяет автоматизировать часть процесса разработки программного обеспечения, в~котором не обязательно участие человека, обеспечивая функции непрерывной интеграции. Работает в~сервлет-контейнере, например, Apache Tomcat\cite{wiki:jenkins}.

Инструмент был выбран, так как является бесплатным, имеет хорошую поддержку и~активное сообщество, а~также потому что~у автора данного дипломного проекта имеется положительный опыт использования технологии.

Для проекта было установлено семейтсво плагинов \textit{Blue Ocean}, процесс сборки, тестирования и~доставки описывается на~подмножестве языка \textit{Groovy} -- \textit{Declarative Pipelines}. Большую часть работы на~себя берёт \textit{fastlane}, описанный в~пункте \ref{sec:testing:ci:fastlane}, главными задачами \textit{Jenkins} являются:

\begin{itemize}
	\item мониторинг git репозитория на~предмет наличия обновлений;
	\item клонинг репозитория и~запуск скриптов \textit{fastlane};
	\item сохранение результатов и~артифактов;
	\item оповещение о~статусе работы через Gtihub \gls{api} и~Slack.
\end{itemize}