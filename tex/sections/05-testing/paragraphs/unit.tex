\subsubsection{}
\label{sec:testing:unit:defenition}

Единичное тестирование, или модульное тестирование -— процесс в программировании, позволяющий проверить на корректность единицы исходного кода, наборы из одного или более программных модулей вместе с соответствующими управляющими данными, процедурами использования и обработки.

Идея состоит в том, чтобы писать тесты для каждой нетривиальной функции или метода. Это позволяет достаточно быстро проверить, не привело ли очередное изменение кода к регрессии, то есть к появлению ошибок в уже оттестированных местах программы, а также облегчает обнаружение и устранение таких ошибок. 

Цель модульного тестирования —- изолировать отдельные части программы и показать, что по отдельности эти части работоспособны\cite{wiki:unit}.

Кроме дополнительных гарантий корректности работы программы, модульное тестирование значительно повышает качество кода и архитектуры самого приложения:

\begin{enumerate}
	\item Тесты заставляют отвественнее подходить к проектированию зависимостей класса.
	\item Модульные тесты в понятной любому человеку форме описывают задачи и особенности работы класса или метода.
	\item Тесты позволяют производить рефакторинг с меньшей степенью регрессий.
	\item По классу тестов типа можно быстро оценить масштаб и разнообразие задач, решаемых типом.
	\item Модульное тестирование вдохновляет писать код таким образом, что позже можно легко заменять части программы на альтернативную реализацию, сохраняя при этом корректность работы остальных модулей.
\end{enumerate}
