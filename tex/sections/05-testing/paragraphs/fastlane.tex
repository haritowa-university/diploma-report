\subsubsection{}
\label{sec:testing:ci:fastlane}

\textbf{Fastlane} -- программное обеспечение с открытым исходным кодом, предназначенное для автоматизации взаимодействия с сервисами Apple: Developer Portal и iTunesConnect. \textit{Fastlane} имеет огромное количество инструментов для работы с \textit{Xcode}, \gls{ide} от компании Apple, явялющейся стандартом в индустрии iOS разработки.

Каждый инструмент в \textit{fastlane} называется \textit{lame}, в рамках димпломного проекта использовались следующие \textit{lane}:

\begin{enumerate}
	\item \textit{Match} -- средство генерации и синхронизации сертификатов и провиженов, использующихся для подписи кода и ресурсов iOS приложений. Инструмент создаёт один сертификат для всех разработчиков и синхронизирует его при помощи Git репозитория.
	\item \textit{Scan} -- инструментов для автоматизации тестирования \textit{Xcode} проектов и аггрегации тестовой информации, например процента прокрытия кода тестами, генерации результатов запуска тестов в понятном для остальных сервисов формате, автоматических поиск тестовых схем и конфигураций проекта.
	\item \textit{Gym} -- средство автоматизации собрки \textit{Xcode} проектов, которое берёт на себя обязанности поиска и выбора схемы сборки, генерации архива приложения, выполнения подписи.
	\item \textit{Deliver} -- семейство инструментов для загрузки подписанного архива в \textit{iTunesConnect}, менеджмента мета информации приложения. Инструмент использовался преймущественно для загрузки приложения в \textit{iTunesConnect} и автоматизации выхода бета версий в сервисе \textit{TestFlight}, описанном в пункте \ref{sec:testing:ci:testflight}.
\end{enumerate}