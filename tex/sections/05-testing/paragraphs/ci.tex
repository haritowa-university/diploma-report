\subsubsection{}
\label{sec:testing:ci:ci}

\textbf{Непрерывная интеграция} (\gls{ci}) — это практика разработки программного обеспечения, которая заключается в~слиянии рабочих копий в~общую основную ветвь разработки несколько раз в~день и~выполнении частых автоматизированных сборок проекта для~скорейшего выявления потенциальных дефектов и~решения интеграционных проблем \cite{wiki:ci}. 

Поскольку проект разрабатывается одним человеком, бенефиты подхода \gls{ci} не являются очевидными, однако многие из~них работают и~с коммандой из~одного человека:

\begin{enumerate}
	\item Так как приложение имеет хороший показатель покрытия тестами, благодаря частым интеграциям, аналог теневого тестирования выполняется на~мельчайшие изменения в~приложении.
	\item Правильно сконфигурированная инфраструктура не позволит добавить в~главную ветку проекта код, который является синтаксически невалидным или~нарушает работу тестов.
	\item Имеется постоянная обратная связь, по~которой понятен уровень технического долга проекта: количество предупреждений компилятора, уровень покрытия тестами.
	\item Каждая интеграция добавляет немного информации в~общую статистику, благодаря чему можно позже делать ценные выводы, например, какой из~модулей приложения отнимает больше всего времени разработки, какие тесты относятся к~классу "мерациющих" (зависят от~сторонних факторов и~дают различный результат для~идентичного кода).
	\item Статистика позволяет отследить изменение скорости сборки проекта, что~позже помогает оптимизировать время компиляции.
	\item При~добавлении нового члена команды, система иммет готовый механизм для~автоматического обнаружения ошибок, что~помогает как новому члену команды, так и~существующим.
\end{enumerate}