\subsubsection{}
\label{sec:testing:unit:tdd}

В экстремальных методиках программирования существует \textit{методика разработки через тестирование}, которая подразумевает использование следующего алгоритма про разработке программного обеспечения:

\begin{enumerate}
	\item Написание интерфейса и протокола типа: описание протокола, который дожен будет адоптировать разрабатываемый тип и написание необходимого и достаточного количества кода для успешной кмпиляции программы с новым типом, реализующим протокол.
	\item Написание тестов для типа. На этом этапе все тесты дожны выдавать ошибку, если тесты успешно проходят -- высоки шансы, что такие тесты ничего не тестируют.
	\item Написание необходимого количества кода для удовлетворения условий тестов. Результатом выполнения данного шага является успешное завершение всех разработанных ранее тестов.
	\item При обновлении требований или разработке большой декомпозированной задачи, пункты повторяются заново.
\end{enumerate}

Данный подход хорош тем, что позволяет установить требования для типа и продумать зависимости ещё до разработки, проверить работоспособность тестов при нерабочем коде и постоянно повышать покрытие кода тестами. При разработке текущего дипломного проекта активно применялся рассмотренный выше подход.