\subsubsection{}
\label{sec:testing:tech:rx}

Для решения задачи тестирования кода, использующего в своих интерфейсах \gls{rx}, в мире \textit{RxSwift} используется две библиотеки: \textit{RxTest} и \textit{RxBlocking}.

\textit{RxTest} преймущественно используется для тестирования собственных расширений \gls{observable}: библиотека предоставляет

\begin{itemize}
	\item тестовый планировщик с контролируемым виртуальным временем;
	\item тестовый \gls{observable}, который конструируется из списка событий и времени, когда они должны произойти;
	\item тестовый \gls{observer}, который записывает все значения, которые в него поступают.
\end{itemize}

\textit{RxBlocking} позволяет превратить подписку на любой \gls{observable} в блокирующий метод, который аггрегирует все события из оригинального \gls{observable} и возвращает управление, когда оригинальный поток завершается. С данной библиотекой тестирование реактивных методов становится похожим на тестирование традиционных синхронных тестов, другими словами программист без особых усилий тестирует асинхронный код, что часто является большой сложностью. В рамках разработки ПП было создано несколько тестовых расширений, одно из них представлено в листинге \ref{sec:testing:tech:rx:testasync}.

\begin{code}[h!]
  \lstinputlisting{inc/src/testasync.swift}
   \caption{Метод для тестирования асинхронного кода}
   \label{sec:testing:tech:rx:testasync}
\end{code}

Данный метод позволяет взять любой \gls{observable}, записать требуемое количество его значений, опционально вызвав перед этим тестовый метод. Если \gls{observable} не успевает сгенерировать достаточное количество событий за отведённое время или генерирует ошибку -- метод возврашает пустое значение.