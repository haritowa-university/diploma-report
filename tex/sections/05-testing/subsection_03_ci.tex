\subsection{Описание и~настройка процесса непрерывной интеграции программного кода}
\label{sec:testing:ci}

Для повышения эффективности разработки \gls{pp}, преждевременного выявления ошибок кодирования и~упрощения процесса контроля качеста, при~разработке проекта использовались техники \gls{ci} и~\gls{cd}.

\subsubsection{}
\label{sec:testing:ci:ci}

\textbf{Непрерывная интеграция} (\gls{ci}) — это практика разработки программного обеспечения, которая заключается в~слиянии рабочих копий в~общую основную ветвь разработки несколько раз в~день и~выполнении частых автоматизированных сборок проекта для~скорейшего выявления потенциальных дефектов и~решения интеграционных проблем \cite{wiki:ci}. 

Поскольку проект разрабатывается одним человеком, бенефиты подхода \gls{ci} не являются очевидными, однако многие из~них работают и~с коммандой из~одного человека:

\begin{enumerate}
	\item Так как приложение имеет хороший показатель покрытия тестами, благодаря частым интеграциям, аналог теневого тестирования выполняется на~мельчайшие изменения в~приложении.
	\item Правильно сконфигурированная инфраструктура не позволит добавить в~главную ветку проекта код, который является синтаксически невалидным или~нарушает работу тестов.
	\item Имеется постоянная обратная связь, по~которой понятен уровень технического долга проекта: количество предупреждений компилятора, уровень покрытия тестами.
	\item Каждая интеграция добавляет немного информации в~общую статистику, благодаря чему можно позже делать ценные выводы, например, какой из~модулей приложения отнимает больше всего времени разработки, какие тесты относятся к~классу "мерациющих" (зависят от~сторонних факторов и~дают различный результат для~идентичного кода).
	\item Статистика позволяет отследить изменение скорости сборки проекта, что~позже помогает оптимизировать время компиляции.
	\item При~добавлении нового члена команды, система иммет готовый механизм для~автоматического обнаружения ошибок, что~помогает как новому члену команды, так и~существующим.
\end{enumerate}
\subsubsection{}
\label{sec:testing:ci:cd}

\textbf{Непрерывная доставка} (\gls{cd}) -- подход при разработке программного обеспечения, в котором комманда производит результаты работы (бинарные файлы и иные артифакты) короткими циклами, делая возможным в сжатые сроки (относительно запроса о выпуске) выпустить новую версия приложения. Данный подход помогает уменьшить финансовые затраты, время и риски доставки изменений ПП, позволяя выпускать итарационные обновления на постоянной основе.

Благодаря \gls{cd}, стало возможным комфортное использование сервиса Apple, \textit{Test Flight}, который будет рассмотрен в пункте \ref{sec:testing:ci:testflight}.
\subsubsection{}
\label{sec:testing:ci:testflight}

В качестве платформы для~распространения бета версий прилоежния был выбран сервис \textbf{TestFlight}. Сервис принадлежит компании Apple и~является частью \textit{iTunesConnect}, который ответственнен за дистрибьюцию iOS приложений через AppStore.

К плюсам сервиса можно отнести:

\begin{itemize}
	\item сервис является абсолютно бесплатным;
	\item ipa архив подписывается провиженом типа AppStore;
	\item в~любой момент можно отравить в~магазин приложений тот же архив, что~и~устанавливается на~тестовые утройства;
	\item приложение можно устанавливать на~\num{10000} устройств без необходимости указываться уникальный идентификатор каждого в~сертификате;
	\item сервис тесно интегрирован с~\textit{iTunesConnect};
	\item сервис поддерживается компанией Apple и~имеет хорошую интеграцию с~операционной системой;
	\item имеется возможность пригласить любого человека в~программу бета тестирования без необходимости добавления уникального идентификатора его устройства.
\end{itemize}

В связи с~особенностями подписи кода, диктуемыми компанией Apple, сервис является лучшим решением на~рынке.

\subsubsection{}
\label{sec:testing:ci:fastlane}

\textbf{Fastlane} -- программное обеспечение с~открытым исходным кодом, предназначенное для~автоматизации взаимодействия с~сервисами Apple: Developer Portal и~iTunesConnect. \textit{Fastlane} имеет огромное количество инструментов для~работы с~\textit{Xcode}, \gls{ide} от~компании Apple, явялющейся стандартом в~индустрии iOS разработки.

Каждый инструмент в~\textit{fastlane} называется \textit{lame}, в~рамках димпломного проекта использовались следующие \textit{lane}:

\begin{enumerate}
	\item \textit{Match} -- средство генерации и~синхронизации сертификатов и~провиженов, использующихся для~подписи кода и~ресурсов iOS приложений. Инструмент создаёт один сертификат для~всех разработчиков и~синхронизирует его при помощи Git репозитория.
	\item \textit{Scan} -- инструментов для~автоматизации тестирования \textit{Xcode} проектов и~аггрегации тестовой информации, например процента прокрытия кода тестами, генерации результатов запуска тестов в~понятном для~остальных сервисов формате, автоматических поиск тестовых схем и~конфигураций проекта.
	\item \textit{Gym} -- средство автоматизации собрки \textit{Xcode} проектов, которое берёт на~себя обязанности поиска и~выбора схемы сборки, генерации архива приложения, выполнения подписи.
	\item \textit{Deliver} -- семейство инструментов для~загрузки подписанного архива в~\textit{iTunesConnect}, менеджмента мета информации приложения. Инструмент использовался преймущественно для~загрузки приложения в~\textit{iTunesConnect} и~автоматизации выхода бета версий в~сервисе \textit{TestFlight}, описанном в~пункте \ref{sec:testing:ci:testflight}.
\end{enumerate}
\subsubsection{}
\label{sec:testing:ci:jenkins}

\textbf{Jenkins} -- проект для~непрерывной интеграции с~открытым исходным кодом, написанный на~Java. Был ответвлён от~проекта Hudson, принадлежащего компании Oracle. Распространяется под~лицензией MIT. Позволяет автоматизировать часть процесса разработки программного обеспечения, в~котором не обязательно участие человека, обеспечивая функции непрерывной интеграции. Работает в~сервлет-контейнере, например, Apache Tomcat\cite{wiki:jenkins}.

Инструмент был выбран, так как является бесплатным, имеет хорошую поддержку и~активное сообщество, а~также потому что у автора данного дипломного проекта имеется положительный опыт использования технологии.

Для проекта было установлено семейтсво плагинов \textit{Blue Ocean}, процесс сборки, тестирования и~доставки описывается на~подмножестве языка \textit{Groovy} -- \textit{Declarative Pipelines}. Большую часть работы на~себя берёт \textit{fastlane}, описанный в~пункте \ref{sec:testing:ci:fastlane}, главными задачами \textit{Jenkins} являются:

\begin{itemize}
	\item мониторинг git репозитория на~предмет наличия обновлений;
	\item клонинг репозитория и~запуск скриптов \textit{fastlane};
	\item сохранение результатов и~артифактов;
	\item оповещение о~статусе работы через Gtihub \gls{api} и~Slack.
\end{itemize}

Таким образом, были рассмотрены все необходимы инструменты для~организации \gls{ci} \gls{cd} инфраструктуры приложения. По итогу разработки можно сделать вывод, что~решение настройки процесса непрерывной доставки и~интеграции значительно ускорил процесс разработки.