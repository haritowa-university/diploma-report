\subsubsection {}
\label{sec:analysis:research:mobArch:mvc}

Классическим архитектурным паттером при~разработке для~платформы iOS является \gls{mvc}. Шаблон присваивает объектам в~ПС одну из~трех ролей: модель, представление или~контроллер. Шаблон определяет не только роли объектов в~ПС, но~и~способ взаимодействия объектов друг с~другом. Каждый из~трех типов объектов отделен от~других абстрактными границами и~связывается с~объектами других типов через эти границы. Набор объектов определенного типа \gls{mvc} в~ПС иногда называется слоем, например, слоем модели\cite{apple:mvc}. На~рисунке \ref{sec:analysis:research:mobArch:apple-mvc:image:mvc} представлена связь между объектами паттерна \gls{mvc}.

\begin{figure}[h]
  \centering
    \includegraphics[width=1\textwidth]{inc/img/mvc.png}
  \caption{Связи между объектами MVC}
  \label{sec:analysis:research:mobArch:apple-mvc:image:mvc}
\end{figure}

\emph{Объекты модели} инкапсулируют данные, специфические для~программного средства и~описывают логику изменения этих данных. Для~примера, объект модели может представлять персонажа в~игре или~контакт в~книге контактов. Объекты модели могут иметь связь один ко многим или~многие ко многим с~остальными объектами модели, и~поэтому иногда модельный уровень программного средства представляет собой один или~более объектных графов. В~хорошей реализации паттерна объекты модели не должны иметь связи с~объектами представления, изолируя данные от~ошибок слоя презентации и~действий пользователя: пользовательские действия в~слое презентации, которые создают или~модифицируют данные, влияют на~модель при~помощи промежуточного слоя котроллер и~являются причиной создания или~обновления объектов модели. Когда объект модели изменяется (например, новые данные были получены из~сети), он уведомляет объект контроллера, который, в~свою очередь, обновляет объект представления.

\emph{Объектами представления} являются объекты, которые видит пользователь. Объекты данной группы умеют представлять себя графически на~экране и~реагировать на~действия пользователя. Основной задачей объектов представления является вывод данных из~модели программного средства и~предоставление пользователю возможности редактировать эти данные. Объекты представления часто обобщатся и~используются между программными средствами, хорошим примером такого переиспользования является \gls{uikit}. 

Объекты типа \emph{контроллер} являются прослойкой между моделью и~представлением, уведомляющих представление об изменениях в~модели и~наоборот. Объекты контроллера также могут выполнять настройку и~координирование задач для~ПС и~управлять жизненными циклами других объектов. 