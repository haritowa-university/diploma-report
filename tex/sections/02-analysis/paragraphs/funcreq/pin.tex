\subsubsection{} Функция пароля приложения
\label{sec:analysis:research:funcreq:pin}

Функция пароля приложения должна быть реализована с~учетом следующих требований:
\begin{enumerate}
	\item При~первой инициализации аккаунта на~устройстве, пользователь обязан установить пароль приложения.
	\item Пароль приложения не хранится на~устройстве, однако используется для~генерации части \gls{aes} ключа приложения.
	\item При~установке пароля, клиент формирует результирующий \gls{aes} ключ, половина которого вычисляется из~пароля пользователя, а~вторая половина вычисялется сервером, используя данныые пользователя, публичного ключа и~хэша от~клиентской части \gls{aes} ключа.
	\item Приложение запрашивает ввод пароля на~каждом запуске или~выхода приложения из~фонового состояния.
	\item В~интерфейсе, приложение обязано полностью закрывать собственное содержимое до~правильного ввода пароля.
	\item Серверная часть ответственнена за подсчёт количества неуспешных попыток ввода пароля, а~клиентская, в~свою очередь, за очистку данных клиента при~вводе неправильного пароля более 10 раз подряд.
\end{enumerate}