\subsubsection{} Функция обмена сообщениями
\label{sec:analysis:research:funcreq:messages}

Обмен сообщениями входит в список основных функций разрабатываемого приложения. При реализации данной функции необходимо учесть следующие требования:

\begin{enumerate}
	\item Должна быть возможность одного сообщения другому пользователю.
	\item Пользовательский интерфейс должен предоставлять возможность на одном экране просматривать всю историю сообщений с другим пользователем и отправлть сообщения.
	\item В пользовательском интерфейсе для каждого сообщения должны быть отражены:
	\begin{enumerate}
		\item текст сообщения;
		\item дата отправки сообщения;
		\item имя отправителя сообщения;
		\item статус сообщения (отправлено, доставлено, прочитано).
	\end{enumerate}
	\item Если совершалась попытка отправки сообщения без интернета, сообщение должно быть отправлено повторно при получении соединения;
	\item Если при отправке сообщения возникла ошибка -- на интерфейсе должно отображаться это сообщение с кнопкой повторной досылки.
\end{enumerate}