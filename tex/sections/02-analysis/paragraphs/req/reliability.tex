\subsubsection{} Требования к~надежности
\label{sec:analysis:research:req:reliability}

Надежное функционирование программы должно быть обеспечено выполнением следующих организационно-технических мероприятий:

\begin{itemize}
	\item организация бесперебойного питания;
	\item выполнение рекомендаций Министерства труда и~социальной защиты РБ, изложенных в~Постановлении от~23 марта 2011 г. «Об утверждении Норм времени на~работы по~обслуживанию персональных электронно-вычислитель\-ных машин, организационной техники и~офисного оборудования»;
	\item выполнение требований ГОСТ 31078-2002 <<Защита информации. Испытания программных средств на~наличие компьютерных вирусов>>;
	\item необходимым уровнем квалификации пользователей.
\end{itemize}

Время восстановления после отказа, вызванного сбоем электропитания технических средств (иными внешними факторами), нефатальным сбоем операционной системы, не должно превышать времени, необходимого на~перезагрузку операционной системы и~запуск программы, при~условии соблюдения условий эксплуатации технических и~программных средств. Время восстановления после отказа, вызванного неисправностью технических средств, фатальным сбоем операционной системы, не должно превышать времени, требуемого на~устранение неисправностей технических средств и~переустановки программных средств.

Отказы программы возможны вследствие некорректных действий пользователя при~взаимодействии с~операционной системой. Во избежание возникновения отказов программы по~указанной выше причине следует обеспечить работу конечного пользователя без предоставления ему административных привилегий.