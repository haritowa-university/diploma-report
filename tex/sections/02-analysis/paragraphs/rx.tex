\subsubsection{}
\label{sec:analysis:research:mobArch:rx}

\emph{Reactive Extensions} -- набор средств и~\gls{api}, позволяющие императивным языкам программирования использовать концепции реактивного программирования и~потоки данных независимо от~того, являются ли потоки синхронными или~асинхронными\cite{wiki:rx}.

\gls{rx} объединяет лучшие идеи паттерна \gls{observer}, \textit{Iterable} и~функционального программировния. Парадигма дополняет паттерн \gls{observer}, добавляет набор операторов для~декларативного описания трансформации и~композиции потоков, абстрагируя вопросы синхронизации, потокобезопасности.

Одной из~важнейших гарантий \gls{rx} является порядок, в~котором подписчик получит данные. \gls{observer} обрабатывает каждое событие по~порядку, не более одного за раз, если события успевают поступать быстрее, чем подписчик их потребляет -- события ставятся в~очередь или~откидываются.

Основным примитивом \gls{rx} является \emph{\gls{observable}}. Observable является идеальным способом рассылки асинхронных событий\cite{programming-in-haskell}.

\begin{table}[h!]
\caption{Разные способы получения значений}
\label{theory:archeticture:rx:call}
\centering
\begin{tabularx}{\textwidth}{ |X|X|X| } 
 \hline
  & \emph{Одно значение} & \emph{Несколько значений} \\ 
 \hline
 \emph{Синхронно} & T getData() & Iterable<T> getData() \\ 
 \hline
 \emph{Асинхронно} & Future<T> & Observable<T> getData() \\ 
 \hline
\end{tabularx}
\end{table}

Создатели \gls{rx} предлагают\cite{reactivex:introduction} рассматривать \gls{observable} как эквивалент \textit{Iterable} с~моделью push (данные приходят, а~не запрашиваются потребителем). В~таблице \ref{theory:archeticture:rx:iterable-observable} рассмотрены аналогичные возможности \textit{Iterable} и~\gls{observable}

\begin{table}[h!]
\caption{Сравнение Observable и~Iterable}
\label{theory:archeticture:rx:iterable-observable}
\centering
\begin{tabularx}{\textwidth}{ |X|X|X| } 
 \hline
 \emph{Событие} & \emph{Iterable(pull)} & \emph{Obsevable(push)} \\ 
 \hline
 \emph{Получение данных} & T next() & onNext(T) \\ 
 \hline
 \emph{Ошибка} & throw & onError(Error) \\ 
 \hline
  \emph{Завершение} & !hasNext() & onCompleted() \\ 
 \hline
\end{tabularx}
\end{table}

\gls{observable} принято разделять на~\emph{холодные} и~\emph{горячие}. Различие заключается в~том, когда \gls{observable} начинает высылать значения, когда выполняется логика подписки. Холодный \gls{observable} запускает работу при~каждой подписке, следовательно высылает полный набор значений каждому подписчику. Горячий \gls{observable} начинает работу (и рассылку событий) при~создании, подписчики получают только события, которые были высланы после подписки, у горячих \gls{observable} нет возможности запустить работу заново. Хорошим примером отличия горячих и~холодных \gls{observable} является сетевой запрос:

\begin{enumerate}
	\item Холодный \gls{observable} будет выполнять запрос при~каждой новой подписке, высылая результат каждому новому подписчику.
	\item Горячий \gls{observable} отправит запрос при~создании, результат получат только те подписчики, которые сформировали подписку до~получения ответа от~сети.
\end{enumerate}

Существуют способы <<охладить>> или~<<подогреть>> \gls{observable}, заставить горячий \gls{observable} высылать определённое количество последних событий всем новым подписчикам, отложить его работу до~вызова метода \textit{connect}.

\gls{observable} и~\gls{observer} являются важными компонентами \gls{rx}, но~не единственными. Сами по~себе они не делают ничего, кроме небольшого расширения привычного паттерна \gls{observer}, адаптированного для~работы с~последовательностью событий. Настоящая сила \gls{rx} приходит с~так называемыми реактивными расширениями: операторами, которые позволяют трансформировать, комбинировать и~манипулировать значениями, произведёнными \gls{observable}. Эти расширения позволяют декларативно компоновать асинхронные последовательности с~высокой производительностью колбеков, но~без побочных эффектов в~виде высокой вложенности и~кода-лапши.

\emph{Оператор} -- функция, которая принимает \gls{observable} своим первым аргументом и~возвращает другой \gls{observable}, который является результатом некоторой трансформации элементов из~первого \gls{observable}.