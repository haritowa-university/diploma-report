\subsubsection{}
\label{sec:analysis:research:crypto:rsa}

\emph{RSA} (аббревиатура от~фамилий Rivest, Shamir и~Adleman) — криптографический алгоритм с~открытым ключом, основывающийся на~вычислительной сложности задачи факторизации больших целых чисел. Криптосистема RSA стала первой системой, пригодной и~для~шифрования, и~для~цифровой подписи. Алгоритм используется в~большом числе криптографических приложений, включая PGP, S/MIME, TLS/SSL, IPSEC/IKE и~других\cite{wiki:rsa}.

Криптографические системы с~открытым ключом используют так называемые односторонние функции, которые обладают следующим свойством:
\begin{itemize}
	\item если известно \(x\), то \(f(x)\) вычислсить относительно просто;
	\item если известно \(y=f(x)\), то для~вычисления \(x\) нет простого (эффективного) пути.
\end{itemize}

RSA-ключи генерируются следующим образом:
\begin{enumerate}
	\item Выбираются два различных случайных простых числа \(q\) и~\(p\) заданного размера.
	\item Вычисляется их произведение \(n=q \cdot p\), которое называется \emph{модулем}.
	\item Вычисляется значение функции Эйлера от~числа \(n\): \(\varphi(n)=(p-1) \cdot (q-1)\).
	\item Выбирается целое число \(e (1 < e < \varphi(n))\), взаимно простое со значением функции \(\varphi(n)\).
	\begin{enumerate}
		\item Число \(e\) называется открытой экспонентой.
		\item Время, необходимое для~шифрования с~использованием быстрого возведения в~степень, пропорционально числу единичных бит в~\(e\).
		\item Слишком малые значения \(e\), например 3, потенциально могут ослабить безопасность схемы RSA.
	\end{enumerate}
	\item Вычисляется число \(d\), мультипликативно обратное к~числу \(e\) по~модулю  \(\varphi(n)\), то есть число, удовлетворяющее сравнению: \(d \cdot e \equiv 1\).
	\item Пара \({e,n}\) публикуется в~качестве открытого ключа RSA.
	\item Пара \({d,n}\)  играет роль закрытого ключа RSA.
\end{enumerate}

Имея публичный ключ, алгоритм шифрования выглядит следующим образом:
\begin{enumerate}
	\item Взять открытый ключ \((e,n)\).
	\item Взять открытый текст \(m\).
	\item Зашифровать сообщение с~использованием открытого ключа: \(c=E(m)=m^e \mod n\)
\end{enumerate}

Имея приватный ключ, алгоритм расшифрования выглядит следующим образом:
\begin{enumerate}
	\item Принять зашифрованное сообщение \(c\).
	\item Взять свой закрытый ключ \((d,n)\).
	\item Применить закрытый ключ для~расшифрования сообщения: \(m=D(c)=c^d \mod n\).
\end{enumerate}