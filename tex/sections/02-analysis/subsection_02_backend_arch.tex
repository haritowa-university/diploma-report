\subsection{Анализ архитектурных подходов и~технологий для построения серверных программных средств}
\label{sec:analysis:research:backArch}

Современные серверные программные средства предполагают использование архитектуры и инструментов, позволяющие

\begin{itemize}
\item получить высокую тестируемость ПС;
\item быстро и стабильно разворачивать всю инфраструктуру ПС;
\item с минимальными затратами масштабировать ПС;
\item получить надёжность и отказоустойчивость.
\end{itemize}

Существуют два основных подхода при разработке серверных программных средств, которые будут рассмотрены в следующих параграфах: монолит и микросервисы.

\subsubsection {}
\label{sec:analysis:research:backArch:monolith}

Монолитная архитектура предполагает развёртывание программног средства одним файлом (например, WAR фаил) или архивом файлов (например, программа на Rails), все модули ПС разрабатываются, развёртываются и тестируются одновременно. Обычно при реализации данного подхода, весь код программног средства использует одну базу данных. На рисунке \ref{sec:analysis:research:arch:back:monolith} представлен пример организации монолитного ПС\cite{microservices:ma}.

\begin{figure}[h]
  \centering
    \includegraphics[width=1\textwidth]{inc/img/backend-monolith.jpg}
  \caption{Пример организации монолитного серверного ПС}
  \label{sec:analysis:research:arch:back:monolith}
\end{figure}

К плюсам подхода можно отнести:

\begin{enumerate}
	\item \emph{Простота в разработке} -- целью современных \gls{ide} является поддержка монолитных приложений.
	\item \emph{Простота в развёртывании} -- для старта работы всего программног средства нужно лишь запустить один процесс с подготовленным окружением.
	\item \emph{Простота в масштабируемости} -- ПС масштабируется путём запуска дополнительных процессов, организованных при помощи балансировщика нагрузки.
\end{enumerate}

Данный подход удобен до определённого размера программного средства, однако, чем больше и сложнее становится программное средство, тем существенней становятся следующие проблемы:

\begin{enumerate}
	\item \emph{Сложность поддержки} -- с увеличением кодовой базы, увеличивается сложность ввода нового специалиста в проект. Результатом является общее замедление разработки и отсутствие возможности быстро ускорить разработку при помощи привлечения дополнительных специалистов.
	\item \emph{Перегруженная \gls{ide}} -- хотя современные \gls{ide} и ориентируеются на монолитные ПС, большая кодовая база способна сильно замедлить работу \gls{ide}.
	\item Долгий запуск контейнера программного средства.
	\item \emph{Сложности при масштабировании} -- на определённом этапе, в коде появляются слабые точки производительности. Часто таких точек несколько и они трeбуют разные виды ресурсов (база, процессор, память), однако единственный способ масштабировать программное средство -- запускать новый процесс со всем ПС внутри, что не позволяет точечно устранять проблемы.
	\item \emph{Привязанность к технологическому стеку} -- монолитное ПС крайне сложно постепенно переводить на новый технологический стек, а единовременное полное портирование ПС -- опасный процесс, производящий огромное количество ошибок.
\end{enumerate}
\input{sections/02-analysis/paragraphs/back-micro-servicies}

Таким образом, были рассмотрены основные архитектурные подходы при разработке серверной части. В разделе сравнены два главных способа организовывать серверные ПС: монолит и миросервисы. По результатам сравнения подходов, было принято решение использовать монолитную архитектуру в текущем дипломном проекте.