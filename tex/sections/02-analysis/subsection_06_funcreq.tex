\subsection{Определение функциональных требований к~разрабатываемому~программному~средству}
\label{sec:analysis:research:funcreq}

С учётом требований, определённых в подразделе \ref{sec:analysis:research:req}, представим детализацию функций проектируемого \gls{pp}.

\subsubsection{} Функция регистрации
\label{sec:analysis:research:funcreq:registration}


Функция регистрации должна быть реализована с учетом следующих требований:

\begin{enumerate}
	\item Процесс регистрации инициируется запуском приложения;
	\item Для регистрации пользователь обязан предоставить номер мобильного телефона.
	\item Правильность предоставленного номера телефона должна проверяться путем отправки сообщения с кодом, ввод которого означает подтверждение пользователя.
	\item В хранении пароля нет необходимости, однако при авторизации/регистрации, устройство создаёт необходимые для работы криптографии файлы.
	\item Должна быть предусмотрена возможность смены пароля.
\end{enumerate}
\subsubsection{} Функция пароля приложения
\label{sec:analysis:research:funcreq:pin}

Функция пароля приложения должна быть реализована с учетом следующих требований:
\begin{enumerate}
	\item При первой инициализации аккаунта на устройстве, пользователь обязан установить пароль приложения.
	\item Пароль приложения не хранится на устройстве, однако используется для генерации части \textit{AES} ключа приложения.
	\item При установке пароля, клиент формирует результирующий \textit{AES} ключ, половина которого вычисляется из паролья пользователя, а вторая половина вычисялется сервером, используя данныые пользователя, публичного ключа и хэша от клиентской части \textit{AES} ключа.
	\item Приложение запрашивает ввод пароля на каждом запуске или выхода приложения из фонового состояния.
	\item Приложение обязано полностью закрывать собственное содержимое до правильного ввода пароля.
	\item Серверная часть ответственнена за подсчёт количества неуспешных попыток ввода пароля, а клиентская, в свою очередь, за очистку данных клиента при вводе неправильного пароля более 10 раз подряд.
\end{enumerate}
\subsubsection{} Функция аутентификации
\label{sec:analysis:research:funcreq:auth}

Функция аутентификации должна быть реализована с~учетом следующих требований:

\begin{enumerate}
	\item Инициатором является приложение, аутентификация требуется на~запуске приложения или после принудительного выхода из~аккаунта.
	\item Должна быть реализована возможность повторной аутентификации пользователя без необходимости ввода какой-либо информации.
	\item Аутентификация должна верифицировать клиента при помощи хэша от~\textit{PEM} устройства.
	\item Если у пользователя уже существуют устройства -- текущее устройство должно подтверждаться с~любого другого активированного.
\end{enumerate}
\subsubsection{} Функция контактов
\label{sec:analysis:research:funcreq:contacts}

В приложении должна быть реализована упрощённая для~пользователя функция контактов, которая удовлетворяет следующим требованиям:

\begin{enumerate}
	\item В~приложении должнен быть отдельный экран просмотра всего списка контактов.
	\item Контактами пользователя являются все контакты из~списка контактов мобильной книги телефона, которые зарегестрированы в~сервисе.
	\item При~установке приложения у пользователя запрашивается доступ к~списку контактов, который отправляется на~сервер.
\end{enumerate}
\subsubsection{} Функция просмотра списка диалогов
\label{sec:analysis:research:funcreq:dialogueslist}

При реализации функции просмотра диалогов, должны быть учтены следующие требования:

\begin{enumerate}
	\item Для вывода списка диалогов в приложении должен быть отдельный экран, который находится в главной навигации.
	\item Список диалогов является остортированным по последнему полученному сообщению списком сгруппированных по признаку идентификатора диалога сообщений.
	\item Каждый элемент списка должен показывать название диалога, и информацию по последнему полученному сообщению:
	\begin{enumerate}
		\item текст последнего сообщения;
		\item время получения последнего сообщения;
		\item статус прочтения сообщения;
		\item имя отправителя сообщения.
	\end{enumerate}
	\item Список диалогов должен динамически обновляться при получении или отправке нового сообщения.
\end{enumerate}

\subsubsection{} Функция обмена сообщениями
\label{sec:analysis:research:funcreq:messages}

Обмен сообщениями входит в~список основных функций разрабатываемого приложения. При реализации данной функции необходимо учесть следующие требования:

\begin{enumerate}
	\item Должна быть возможность одного пользователя отправить сообщение другому пользователю.
	\item Пользовательский интерфейс должен предоставлять возможность на~одном экране просматривать всю историю сообщений с~другим пользователем и~отправлть сообщения.
	\item В~пользовательском интерфейсе для~каждого сообщения должны быть отражены:
	\begin{enumerate}
		\item текст сообщения;
		\item дата отправки сообщения;
		\item имя отправителя сообщения;
		\item статус сообщения (отправлено, доставлено, прочитано).
	\end{enumerate}
	\item Если совершалась попытка отправки сообщения без интернета, сообщение должно быть отправлено повторно при получении соединения.
	\item Если при отправке сообщения возникла ошибка -- на~интерфейсе должно отображаться это сообщение с~кнопкой повторной досылки.
\end{enumerate}

По итогам текущего раздела были сформулированны функциональные требования, наличие которых позволяет перейти к стадии технического проектирования программного продукта.

% \section[this is a very long title I want to break manually]{\texorpdfstring{this is a very long title I\\ want to break manually}{this is a very long title I want to break manually}}


