\subsubsection {}
\label{sec:analysis:research:mobArch:mvvm}

Шаблон \gls{mvvm} применяется при проектировании архитектуры приложения и позволяет разделить модель, представление и логику её представления. Первоначально был представлен сообществу Джоном Госсманом (John Gossman) в 2005 году как модификация шаблона Presentation Model. \cite{wiki:mvvm}

\gls{mvvm} является альтернативой паттерну \gls{mvc}. С паттерном \gls{mvvm} тесно связан паттерн <<связывание данных>>, позволяющий задать двухстороннюю связь между данными и отображением(сильно упрощая решение задачи динамически обновляемого контента). Шаблон состоит из трёх частей, связанных определённым образом: Model, View, ViewModel.

\emph{Model}(Модель) представляет собой бизнес логику и фундаментальные данные, необходимые для работы приложения.

\emph{View}(Представление) --- графический интерфейс, то есть окно, кнопки и так продолжая. Представление является подписчиком на события изменений значений свойств и пользователем команд, предоставляемых Моделью Представления.

\emph{ViewModel}(Модель Представления) является прослойкой между Моделью и Представлением, предоставляя подготовленные для связывания данные из Модели и набор команд для изменения Модели Представлением.

Шаблон отвечает всем поставленным требованиям: является лёгким в использовании, позволяет разделить интерфейс на независимые компоненты, которые в будущем могут быть доработаны без необходимости дополнительных изменений в Представлении и имеет паттерны для динамического обновления данных. Исходя из этого было принято решение выбрать \gls{mvvm} в качестве основы для будущей архитектуры пользовательского интерфейса приложения.