\subsubsection{}
\label{sec:development:arch:ios:modules}

Финальным пунктом в обзоре архитектуры будущего приложения является выделение основных модулей.

\begin{enumerate}
	
	\item Foundation.Core:
	\begin{enumerate}
		\item Криптография.
		\item База данных.
		\item Веб клиент.
		\item Файловый менеджер.
		\item Менеджер паролей(для работы с KeyChain).
		\item Websocket клиент.
		\item Расширения сторонних библиотек, например RxSwift.
	\end{enumerate}

	\item Foundation.UI:
	\begin{enumerate}
		\item Имплементация MVVM, биндингов.
		\item Переиспользуемые контролы приложения.
		\item Расширения стандартной и сторонних библиотек пользовательского интерфейса, например RxCocoa, UIKit.
		\item Наборы стилей, шрифтов, цветов и ресурсов.
		\item Строки локализации.
	\end{enumerate}

	\item Product.Authorization:
	\begin{enumerate}
		\item Все экраны авторизации.
		\item Модуль авторизации, который предсавляет из себя граф состояний авторизации и рёбра переходов.
		\item \gls{observable} текущего состояния авторизации и данных пользователя.
		\item Веб клиент, имплементирующий \gls{api} атворизации.
		\item Декоратор стандартного веб клиента и websocket клиента, интегрирующий авторизацию в отправляемые запросы.
		\item Декораторы для остальных сервисов, требующих авторизации для полноценной работы(база данных, файловое хранилище).
	\end{enumerate}

	\item Product.Contacts:
	\begin{enumerate}
		\item Экран списка контактов и информации по конкретному контакту.
		\item Сервис для работы с базой данных контактов.
		\item Адаптер, обрабатывающий и посылающий сообщения в websocket, связанные с контактами и списком устройств.
	\end{enumerate}

	\item Product.Dialogues:
	\begin{enumerate}
		\item Экран списка диалогов и деталей по конкретному диалогу.
		\item Сервис для работы с базой данных диалогов.
		\item Имплементация веб клиента для работы с \gls{api} списка диалогов.
	\end{enumerate}

	\item Product.Chat:
	\begin{enumerate}
		\item Экран диалога.
		\item Сервис для работы с базой данных сообщений.
		\item Адаптер, обрабатывающий и посылающий сообщения в websocket, связанные с сообщениями.
		\item Имплементация веб клиента для работы с \gls{api} списка сообщений.
	\end{enumerate}

	\item Product.Settings:
	\begin{enumerate}
		\item Экран настроек.
	\end{enumerate}

	\item App --- готовое приложение, в котором кооржинируется настройка и работа всех модулей.

\end{enumerate}

Как видно из этого списка, каждый модуль требует тесного общения с остальными модулями из своего домена, однако паттерны адавтер и декоратор позволят наслаивать возможности на готовые сервисы, не связывая модули друг с другом.