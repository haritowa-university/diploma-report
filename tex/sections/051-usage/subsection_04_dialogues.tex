\subsection{Просмотр списка диалогов}
\label{sec:usage:dialogues}

После разблокировки клиента, пользователь попадает в главный экран приложения, который показывает список диалогов \ref{sec:usage:dialogues:list}. В список диалогов попадают все диалоги, которые имеют хотя бы одно сообщение. Диалоги сортируются по дате последнего сообщения, открыть конкретный диалог можно нажав на ячейку с его последним сообщением в таблице. Для удаления диалога, нужно потянуть ячейку диалога влево и нажать кнопку <<Удалить>>.

\subsection{Просмотр сообщений диалога и отправка сообщения}
\label{sec:usage:dialogue}

После выбора диалога, пользователь попадает в экран списка сообщений конкретного диалога \ref{sec:usage:dialogues:single}. На экране выводится список сообщений, отсортированный в хронологическом порядке. У каждого сообщения есть статус, который представлен в пользовательском интерфейсе в виде кружка в правом нижнем углу ячейки сообщения. В таблице \ref{table:usage:dialogues:statusdesc} представлены возможные варианты статуса сообщения и соответствующее им состояния интерфейса.

\begin{enumerate}
	\item \textit{Отправляется} -- сообщение отправляется на сервер
	\item доставлено;
	\item прочитано;
\end{enumerate}

Для отправки сообщения, пользователь должен ввести текст в поле <<Сообщение>> и нажать кнопку <<Отправить>>.

\begin{figure}[H]
\centering
\begin{minipage}{.5\textwidth}
  \centering
  \includegraphics[height=0.25\textheight]{inc/img/ui/dialogues_unread.png}
  \captionof{figure}{Экран списка диалогов}
  \label{sec:usage:dialogues:list}
\end{minipage}%
\begin{minipage}{.5\textwidth}
  \centering
  \includegraphics[height=0.25\textheight]{inc/img/ui/single_dialogue.png}
  \captionof{figure}{Экран диалога}
  \label{sec:usage:dialogues:single}
\end{minipage}
\end{figure}

\begin{table}[!ht]
  \caption{Описание статусов сообщения}
  \label{table:usage:dialogues:statusdesc}
  \centering
  \begin{tabularx}{\linewidth}{
    |>{\hsize=0.75\hsize}X|
    |>{\hsize=1\hsize}X|
    >{\centering\arraybackslash\hsize=1.25\hsize}X|
  }
	\hline
	\begin{center}Название статуса\end{center} & Описание статуса & Описание интерфейса \\

	\hline
	Отправляется & сообщение отправляется на сервер & круг с белой заливкой и серой рамкой \\

	\hline
	Отправлено & сообщение отправлено на сервер & круг с белой заливкой и синей рамкой \\

	\hline
	Прочитано & сообщение прочитано получателем & круг с синей заливкой или отсутствие круга, если после текущего сообщения есть другие прочитанные \\

	\hline
  \end{tabularx}
\end{table}