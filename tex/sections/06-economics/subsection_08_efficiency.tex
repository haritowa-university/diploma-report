\subsection{Расчёт показателей эффективности использования программного продукта}
\label{sec:economics:efficiency}

Для определения годового экономического эффекта от разработанной программы необходимо определить суммарные капитальные затраты на разработку и внедрение программы по формуле:
\totalCapitalExpensesEquation

Капитальные и приравненные к ним затраты определяются:
\begin{enumerate}
	\item в случае, если необходимо приобритение нового ПК для решения комплекса задач, в который входит рассматриваемая;
	\item в случае, если ПК, на котором предполагается решать рассматриваемую задачу, отслужил к моменту расчёта х лет.
\end{enumerate}

Так как ни один из приведённых выше случаев не относится к рассматриваемой задаче, \(\capitalExpenses = \capitalExpensesValue\).

Рассчитаем суммарные капитальные затраты на разработку и внедрение программы, подставив значения в формулу \formularef{totalCapitalExpensesEquation}:
\totalCapitalExpensesFormulaApplied

Годовой экономический эффект от сокращения ручного труда при обработке информации определяется по формуле:
\annualEconomicEffectEquation

Рассчитаем годовой экономический эффект от сокращения ручного труда при обработке информации, подставив значения в формулу \formularef{annualEconomicEffectEquation}:
\annualEconomicEffectFormulaApplied

Срок возврата инвестиций определяется по формуле:
\investmentMoneyReturnTimeEquation

Рассчитаем срок возврата инвестиций, подставив значения в формулу \formularef{investmentMoneyReturnTimeEquation}:
\investmentMoneyReturnTimeFormulaApplied