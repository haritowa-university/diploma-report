\subsection{Расчёт показателей эффективности использования программного продукта}
\label{sec:economics:efficiency}

Для определения годового экономического эффекта от разработанной программы необходимо определить суммарные капитальные затраты на разработку и внедрение программы по формуле:
\totalCapitalExpensesEquation

Капитальные и приравненные к ним затраты определяются:
\begin{enumerate}
	\item в случае, если необходимо приобритение нового ПК для решения комплекса задач, в который входит рассматриваемая;
	\item в случае, если ПК, на котором предполагается решать рассматриваемую задачу, отслужил к моменту расчёта х лет.
\end{enumerate}

Так как ни один из приведённых выше случаев не относится к рассматриваемой задаче, \(\capitalExpenses = \capitalExpensesValue\).

Рассчитаем суммарные капитальные затраты на разработку и внедрение программы, подставив значения в формулу \formularef{totalCapitalExpensesEquation}:
\totalCapitalExpensesFormulaApplied

Годовой экономический эффект от сокращения ручного труда при обработке информации определяется по формуле:
\annualEconomicEffectEquation

Рассчитаем годовой экономический эффект от сокращения ручного труда при обработке информации, подставив значения в формулу \formularef{annualEconomicEffectEquation}:
\annualEconomicEffectFormulaApplied

Срок возврата инвестиций определяется по формуле:
\investmentMoneyReturnTimeEquation

Рассчитаем срок возврата инвестиций, подставив значения в формулу \formularef{investmentMoneyReturnTimeEquation}:
\investmentMoneyReturnTimeFormulaApplied

Результаты расчётов приведены в таблице \ref{table:economics:effect:initial_data}.

\begin{table}[!ht]
  \caption{Технико-экономические показатели проекта}
  \label{table:economics:effect:initial_data}
  \centering
  \begin{tabularx}{\linewidth}{
    |>{\hsize=1.6\hsize}X|
    >{\centering\arraybackslash\hsize=0.4\hsize}X|
  }
	\hline
	\begin{center}Наименование показателя\end{center} & Величина показателя \\

	\hline
	1. Трудоёмкость разработки ПП, ч & \num{\totalPowerConsumptionDevelopment} \\

	\hline
	2. Периодичность решения задачи, раз/год & \num{\manualExecutionRateValue} \\

	\hline
	3. Годовые текущие затраты, связанные с решением задачи, \ye & \num{\manualExpensesValue} \\

	\hline
	4. Отпускная цена программы, \ye & \num{\ppResultCostValue} \\

	\hline
	5. Степень новизны программы &  \\

	\hline
	6. Группа сложности алгоритма &  \\

	\hline
	7. Прирост условной прибыли, \ye & \num{\incomeBoostValue} \\

	\hline
	8. Годовой экономический эффект, \ye & \num{\annualEconomicEffectValue} \\

	\hline
	9. Срок возврата инвестиций, лет & \num{\investmentMoneyReturnTimeValue} \\

	\hline
  \end{tabularx}
\end{table}

Таким образом было произведено технико-экономическое обоснование разрабатываемого проекта, составлена смета затрат, рассчитана прогнозируемая прибыль и показана экономическая целесообразность разработки.

В результате были получены следующие значения показателей:
\begin{itemize}
	\item прогнозируемая отпуская цена составляет \num{\ppResultCostValue} \ye;
	\item годовой экономический эффект равен \num{\annualEconomicEffectValue} \ye;
	\item срок возврата инвестиций рвен \num{\investmentMoneyReturnTimeValue} лет.
\end{itemize}

Полученные результаты свидетельствуют об эффективности разработки и внедрения в эксплуатацию разработанного программного приложения.
