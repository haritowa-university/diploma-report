\FPeval{\powerConsumptionTaskDefinitionDesign}{2}
\FPeval{\powerConsumptionTaskDefinitionDevelopment}{1}
\FPeval{\powerConsumptionTaskDefinitionTesting}{1}

\FPeval{\powerConsumptionCodingDesign}{0}
\FPeval{\powerConsumptionCodingDevelopment}{23}
\FPeval{\powerConsumptionCodingTesting}{5}

\FPeval{\pcEnergyConsumptionValue}{0.25}
\FPeval{\energyConsumptionCostValue}{0.25}

\FPeval{\pcPcCostValue}{1500}
\FPeval{\priceIncreaseCoefValue}{1}
\FPeval{\mountCoefValue}{1.05}
\FPeval{\amortizationNormValue}{10}

\FPeval{\pcMaintainanceCoefValue}{0.13}

\FPeval{\devPlaceMeterCostValue}{300}
\FPeval{\additionalDevPlaceCoefValue}{3}
\FPeval{\pcConsumedPlaceValue}{1}
\FPeval{\amortizationDevPlaceNormValue}{1.2}

\FPeval{\devPlaceSupportCoefValue}{0.05}

\FPeval{\rentNormValue}{13.6}
\FPeval{\devPlaceCommfrotCoefValue}{0.9}
\FPeval{\devPlaceGeoEncreaseCoefValue}{0.81}

\FPeval{\pcWorkloadValue}{7}
\FPeval{\pcAverageUptimeValue}{207}



\FPeval{\bonuceRateCoefValue}{0.3}
\FPeval{\additionalSalaryCoefValue}{0.15}
\FPeval{\salaryFundCoefValue}{0.346}

\FPeval{\softwareDebugComplexityValue}{14}
\FPeval{\plannignAdditionalExpensesCoefValue}{1.15}

\FPeval{\projectIncomeNormValue}{0.25}

\FPeval{\ndsValue}{0.2}



\FPeval{\singleManualExecutionEffortValue}{0.085}
\FPeval{\manualExecutionRateValue}{5040}


\FPeval{\averageSymbolsPerInputValue}{75}
\FPeval{\symbolsSpeedNormValue}{1}
\FPeval{\calculationTimeValue}{0.08}
\FPeval{\manualOutputTimeValue}{0.01}
\FPeval{\preparationTimeCoefValue}{0.1}



\FPeval{\incomeTaxValue}{0.18}
\FPeval{\capitalExpensesValue}{0}
\FPeval{\effeciencyCoefValue}{0.4}



\FPeval{\userAverageSalaryFirstDegreeValue}{700}
\FPeval{\averageSalaryFirstDegreeValue}{700}
\FPeval{\monkeyAverageSalaryFirstDegreeValue}{700}


\FPeval{\degreeSalaryCoefValue}{3.34}
\FPeval{\degreeMonkeSalaryCoefValue}{3.34}
\FPeval{\degreeUserSalaryCoefValue}{2.97}


% Расчет трудоёмкости разработки программного продукта
% Calculations
\ADD{\powerConsumptionTaskDefinitionDesign}{\powerConsumptionTaskDefinitionDevelopment}{\temp}
\ADD{\temp}{\powerConsumptionTaskDefinitionTesting}{\temp}
\ADD{\temp}{\powerConsumptionCodingDesign}{\temp}
\ADD{\temp}{\powerConsumptionCodingDevelopment}{\temp}
\ADD{\temp}{\powerConsumptionCodingTesting}{\temp}
\MULTIPLY{\temp}{8}{\totalPowerConsumptionDevelopment}

% Formulas
\newcommand{\powerConsumptionDevelopmentFormula}[1]{%
\powerConsumptionDevelopment = (\sum_{i=1}^{#1} \powerConsumptionTaskDefinitionI + \sum_{i=1}^{#1} \powerConsumptionCodingI) \cdot 8%
}

\newcommand{\powerConsumptionDevelopmentFormulaApplied}{%
\begin{center}
\(\powerConsumptionDevelopmentFormula{3} = (\powerConsumptionTaskDefinitionDesign + \powerConsumptionTaskDefinitionDevelopment + \powerConsumptionTaskDefinitionTesting + \powerConsumptionCodingDesign + \powerConsumptionCodingDevelopment + \powerConsumptionCodingTesting) \cdot 8 = \totalPowerConsumptionDevelopment \, \days\)
\end{center}
}

\newcommand{\powerConsumptionDevelopmentEquation}{%
\begin{equation}\label{powerConsumptionDevelopmentEquation}
\powerConsumptionDevelopmentFormula{n}
\end{equation}
\begin{explanation}
где & $ n $ & количество этапов разработки программы; \\
    & $ \powerConsumptionTaskDefinitionI $ & трудоёмкость постановки задачи на \(i\)-м этапе разработки программы, дней; \\
    & $ \powerConsumptionCodingI $ & трудоёмкость программирования задачи на \(i\)-м этапе разработки программы, дней.
\end{explanation}
}



% Расчет стоимости машино-часа работы ПК
%% Energy hour price
\MULTIPLY{\pcEnergyConsumptionValue}{\energyConsumptionCostValue}{\electricityCostValue}

\newcommand{\energyHourCostFormula}{
\electricityCost = \pcEnergyConsumption \cdot \energyConsumptionCost
}

\newcommand{\energyHourCostEquation}{
\begin{equation}\label{energyHourCostEquation}
\energyHourCostFormula,
\end{equation}
\begin{explanation}
где & $ \pcEnergyConsumption $ & среднечасовое потребление электроэнергии \pc, \kvtShort; \\
    & $ \energyConsumptionCost $ & стоимость киловатт-часа электроэнергии, \ye
\end{explanation}
}

\newcommand{\energyHourCostFormulaApplied}{%
\begin{center}
\(\electricityCost = \pcEnergyConsumptionValue \cdot \energyConsumptionCostValue = \electricityCostValue \, \ye\)
\end{center}
}



%% Amortization PC renovation cost
\MULTIPLY{\pcCostValue}{\priceIncreaseCoefValue}{\temp}
\MULTIPLY{\temp}{\mountCoefValue}{\temp}
\COPY{\temp}{\pcBalanceCostResult}
\MULTIPLY{\temp}{\amortizationNormValue}{\temp}
\MULTIPLY{\temp}{0.01}{\pcAmortizationCostResult}
\ROUND[1]{\pcAmortizationCostResult}{\pcAmortizationCostResult}

\newcommand{\pcAmortizationCostFormula}{
\pcAmortizationCost = \pcCost \cdot \priceIncreaseCoef \cdot \mountCoef \cdot \frac{\amortizationNorm}{100} = \pcCostBalance \cdot \frac{\amortizationNorm}{100}
}

\newcommand{\pcAmortizationCostEquation}{
\begin{equation}\label{pcAmortizationCostEquation}
\pcAmortizationCostFormula,
\end{equation}
\begin{explanation}
где & $ \pcCost $ & цена \pc на момент его выпуска, \ye; \\
    & $ \priceIncreaseCoef $ & коэффициент удорожания \pc; \\
    & $ \mountCoef $ & коэффициент, учитывающий затраты на монтаж и транспортировку \pc; \\
    & $ \pcCostBalance $ & балансовая стоимость \pcye; \\
    & $ \amortizationNorm $ & норма амортизационных отчислений на \pc.
\end{explanation}
}

\newcommand{\pcAmortizationCostFormulaApplied}{%
\begin{center}
\(\pcAmortizationCost = \pcCostValue \cdot \priceIncreaseCoefValue \cdot \mountCoefValue \cdot \frac{\amortizationNormValue}{100} = \pcAmortizationCostResult \, \ye\)
\end{center}
}



%% PC Support price
\MULTIPLY{\pcBalanceCostResult}{\pcMaintainanceCoefValue}{\pcRepairCostResult}

\newcommand{\pcSupportCostFormula}{
\pcRepairCost = \pcCostBalance \cdot \pcMaintainanceCoef
}

\newcommand{\pcSupportCostEquation}{
\begin{equation}\label{pcSupportCostEquation}
\pcSupportCostFormula,
\end{equation}
\begin{explanation}
где & $ \pcCostBalance $ & балансовая стоимость \pcye; \\
    & $ \pcMaintainanceCoef $ & коэффициент, учитывающий затраты на ремонт и техническое обсолуживание \pc, в том числе затраты на запчасти, зарплату ремонтного персонала и др.
\end{explanation}
}

\newcommand{\pcSupportCostFormulaApplied}{%
\begin{center}
\(\pcRepairCost = \pcCostValue \cdot \priceIncreaseCoefValue \cdot \mountCoefValue \cdot \pcMaintainanceCoefValue = \pcRepairCostResult \, \ye\)
\end{center}
}



%% PC hour cost
\newcommand{\pcHourCostFormula}{%
\cpuTimeCost = \electricityCost + \frac{\pcAmortizationCost + \pcRepairCost + \devPlaceAmortizationCost + \devPlaceRepairCost + \devPlaceRentCost}{\pcTimeFound}
}

\newcommand{\pcHourCostEquation}{
\begin{equation}\label{pcHourCostEquation}
\pcHourCostFormula,
\end{equation}
\begin{explanation}
где & $ \electricityCost $ & расходы на электроэнергию за час работы \pcye; \\
    & $ \pcAmortizationCost $ & годовая величина амортизационных отчислений на реновацию \pcye; \\
    & $ \pcRepairCost $ & годовые затраты на ремонт и техническое обслуживание \pcye; \\
    & $ \devPlaceAmortizationCost $ & годовая величина амортизационных отчислений на реновацию производственных площадей, \ye; \\
    & $ \devPlaceRepairCost $ & годовые затраты на ремонт и содержание производственных площадей, \ye; \\
    & $ \devPlaceRentCost $ & годовая величина арендных платежей за помещение, занимаемое \pcye; \\
    & $ \pcTimeFound $ & годовой фонд времени работы \pcye.
\end{explanation}
}