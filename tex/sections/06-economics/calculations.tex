\COPY{2}{\powerConsumptionTaskDefinitionDesign}
\COPY{1}{\powerConsumptionTaskDefinitionDevelopment}
\COPY{1}{\powerConsumptionTaskDefinitionTesting}

\COPY{0}{\powerConsumptionCodingDesign}
\COPY{25}{\powerConsumptionCodingDevelopment}
\COPY{5}{\powerConsumptionCodingTesting}

\COPY{0.25}{\pcEnergyConsumptionValue}
\COPY{0.25}{\energyConsumptionCostValue}

\COPY{1500}{\pcCostValue}
\COPY{1}{\priceIncreaseCoefValue}
\COPY{1.05}{\mountCoefValue}
\COPY{10}{\amortizationNormValue}

% Расчет трудоёмкости разработки программного продукта
% Calculations
\ADD{\powerConsumtionTaskDefinitionDesign}{\powerConsumtionTaskDefinitionDevelopment}{\temp}
\ADD{\temp}{\powerConsumtionTaskDefinitionTesting}{\temp}
\ADD{\temp}{\powerConsumtionCodingDesign}{\temp}
\ADD{\temp}{\powerConsumtionCodingDevelopment}{\temp}
\ADD{\temp}{\powerConsumtionCodingTesting}{\temp}
\MULTIPLY{\temp}{8}{\totalPowerConsumtionDevelopment}

% Formulas
\newcommand{\powerConsumtionDevelopmentFormula}[1]{%
\powerConsumtionDevelopment = (\sum_{i=1}^{#1} \powerConsumtionTaskDefinitionI + \sum_{i=1}^{#1} \powerConsumtionCodingI) \cdot 8%
}

\newcommand{\powerConsumtionDevelopmentFormulaApplied}{%
\begin{center}
\(\powerConsumtionDevelopmentFormula{3} = (\powerConsumtionTaskDefinitionDesign + \powerConsumtionTaskDefinitionDevelopment + \powerConsumtionTaskDefinitionTesting + \powerConsumtionCodingDesign + \powerConsumtionCodingDevelopment + \powerConsumtionCodingTesting) \cdot 8 = \totalPowerConsumtionDevelopment \, \days\)
\end{center}
}

\newcommand{\powerConsumtionDevelopmentEquation}{%
\begin{equation}\label{powerConsumtionDevelopmentEquation}
\powerConsumtionDevelopmentFormula{n}
\end{equation}
\begin{explanation}
где & $ n $ & количество этапов разработки программы; \\
    & $ \powerConsumtionTaskDefinitionI $ & трудоёмкость постановки задачи на \(i\)-м этапе разработки программы, дней; \\
    & $ \powerConsumtionCodingI $ & трудоёмкость программирования задачи на \(i\)-м этапе разработки программы, дней.
\end{explanation}
}


% Расчет стоимости машино-часа работы ПК
%% Energy hour price
\MULTIPLY{\pcEnergyConsumptionValue}{\energyConsumptionCostValue}{\electricityCostValue}

\newcommand{\energyHourCostFormula}{
\electricityCost = \pcEnergyConsumption \cdot \energyConsumptionCost
}

\newcommand{\energyHourCostEquation}{
\begin{equation}\label{energyHourCostEquation}
\energyHourCostFormula,
\end{equation}
\begin{explanation}
где & $ \pcEnergyConsumption $ & среднечасовое потребление электроэнергии \pc, \kvtShort; \\
    & $ \energyConsumptionCost $ & стоимость киловатт-часа электроэнергии, \ye
\end{explanation}
}

\newcommand{\energyHourCostFormulaApplied}{%
\begin{center}
\(\electricityCost = \pcEnergyConsumptionValue \cdot \energyConsumptionCostValue = \electricityCostValue \, \ye\)
\end{center}
}

%% Amortization PC renovation cost
\MULTIPLY{\pcCostValue}{\priceIncreaseCoefValue}{\temp}
\MULTIPLY{\temp}{\mountCoefValue}{\temp}
\MULTIPLY{\temp}{\amortizationNormValue}{\temp}
\MULTIPLY{\temp}{0.01}{\pcAmortizationCostResult}
\ROUND[1]{\pcAmortizationCostResult}{\pcAmortizationCostResult}

\newcommand{\pcAmortizationCostFormula}{
\pcAmortizationCost = \pcCost \cdot \priceIncreaseCoef \cdot \mountCoef \cdot \frac{\amortizationNorm}{100} = \pcCostBalance \cdot \frac{\amortizationNorm}{100}
}

\newcommand{\pcAmortizationCostEquation}{
\begin{equation}\label{pcAmortizationCostEquation}
\pcAmortizationCostFormula,
\end{equation}
\begin{explanation}
где & $ \pcCost $ & цена \pc на момент его выпуска, \ye; \\
    & $ \priceIncreaseCoef $ & коэффициент удорожания \pc; \\
    & $ \mountCoef $ & коэффициент, учитывающий затраты на монтаж и транспортировку \pc; \\
    & $ \pcCostBalance $ & балансовая стоимость \pcye; \\
    & $ \amortizationNorm $ & норма амортизационных отчислений на \pc.
\end{explanation}
}

\newcommand{\pcAmortizationCostFormulaApplied}{%
\begin{center}
\(\pcAmortizationCost = \pcCostValue \cdot \priceIncreaseCoefValue \cdot \mountCoefValue \cdot \frac{\amortizationNormValue}{100} = \pcAmortizationCostResult \, \ye\)
\end{center}
}

%% PC hour cost
\newcommand{\pcHourCostFormula}{%
\cpuTimeCost = \electricityCost + \frac{\pcAmortizationCost + \pcRepairCost + \devPlaceAmortizationCost + \devPlaceRepairCost + \devPlaceRentCost}{\pcTimeFound}
}

\newcommand{\pcHourCostEquation}{
\begin{equation}\label{pcHourCostEquation}
\pcHourCostFormula,
\end{equation}
\begin{explanation}
где & $ \electricityCost $ & расходы на электроэнергию за час работы \pcye; \\
    & $ \pcAmortizationCost $ & годовая величина амортизационных отчислений на реновацию \pcye; \\
    & $ \pcRepairCost $ & годовые затраты на ремонт и техническое обслуживание \pcye; \\
    & $ \devPlaceAmortizationCost $ & годовая величина амортизационных отчислений на реновацию производственных площадей, \ye; \\
    & $ \devPlaceRepairCost $ & годовые затраты на ремонт и содержание производственных площадей, \ye; \\
    & $ \devPlaceRentCost $ & годовая величина арендных платежей за помещение, занимаемое \pcye; \\
    & $ \pcTimeFound $ & годовой фонд времени работы \pcye.
\end{explanation}
}