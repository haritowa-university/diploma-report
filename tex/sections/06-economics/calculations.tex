\FPeval{\powerConsumptionTaskDefinitionDesign}{2}
\FPeval{\powerConsumptionTaskDefinitionDevelopment}{1}
\FPeval{\powerConsumptionTaskDefinitionTesting}{1}

\FPeval{\powerConsumptionCodingDesign}{0}
\FPeval{\powerConsumptionCodingDevelopment}{23}
\FPeval{\powerConsumptionCodingTesting}{5}

\FPeval{\pcEnergyConsumptionValue}{0.25}
\FPeval{\energyConsumptionCostValue}{0.25}

\FPeval{\pcPcCostValue}{1500}
\FPeval{\priceIncreaseCoefValue}{1}
\FPeval{\mountCoefValue}{1.05}
\FPeval{\amortizationNormValue}{10}

\FPeval{\pcMaintainanceCoefValue}{0.13}

\FPeval{\devPlaceMeterCostValue}{300}
\FPeval{\additionalDevPlaceCoefValue}{3}
\FPeval{\pcConsumedPlaceValue}{1}
\FPeval{\amortizationDevPlaceNormValue}{1.2}

\FPeval{\devPlaceSupportCoefValue}{0.05}

\FPeval{\rentNormValue}{13.6}
\FPeval{\devPlaceCommfrotCoefValue}{0.9}
\FPeval{\devPlaceGeoEncreaseCoefValue}{0.81}

\FPeval{\pcWorkloadValue}{7}
\FPeval{\pcAverageUptimeValue}{207}



\FPeval{\bonuceRateCoefValue}{0.3}
\FPeval{\additionalSalaryCoefValue}{0.15}
\FPeval{\salaryFundCoefValue}{0.346}

\FPeval{\softwareDebugComplexityValue}{14}
\FPeval{\plannignAdditionalExpensesCoefValue}{1.15}

\FPeval{\projectIncomeNormValue}{0.25}

\FPeval{\ndsValue}{0.2}



\FPeval{\singleManualExecutionEffortValue}{0.085}
\FPeval{\manualExecutionRateValue}{5040}


\FPeval{\averageSymbolsPerInputValue}{75}
\FPeval{\symbolsSpeedNormValue}{1}
\FPeval{\calculationTimeValue}{0.08}
\FPeval{\manualOutputTimeValue}{0.01}
\FPeval{\preparationTimeCoefValue}{0.1}



\FPeval{\incomeTaxValue}{0.18}
\FPeval{\capitalExpensesValue}{0}
\FPeval{\effeciencyCoefValue}{0.4}



\FPeval{\userAverageSalaryFirstDegreeValue}{700}
\FPeval{\averageSalaryFirstDegreeValue}{700}
\FPeval{\monkeyAverageSalaryFirstDegreeValue}{700}


\FPeval{\degreeSalaryCoefValue}{3.34}
\FPeval{\degreeMonkeSalaryCoefValue}{3.34}
\FPeval{\degreeUserSalaryCoefValue}{2.97}


% Расчет трудоёмкости разработки программного продукта
% Calculations
\ADD{\powerConsumtionTaskDefinitionDesign}{\powerConsumtionTaskDefinitionDevelopment}{\temp}
\ADD{\temp}{\powerConsumtionTaskDefinitionTesting}{\temp}
\ADD{\temp}{\powerConsumtionCodingDesign}{\temp}
\ADD{\temp}{\powerConsumtionCodingDevelopment}{\temp}
\ADD{\temp}{\powerConsumtionCodingTesting}{\temp}
\MULTIPLY{\temp}{8}{\totalPowerConsumtionDevelopment}

% Formulas
\newcommand{\powerConsumtionDevelopmentFormula}[1]{%
\powerConsumtionDevelopment = (\sum_{i=1}^{#1} \powerConsumtionTaskDefinitionI + \sum_{i=1}^{#1} \powerConsumtionCodingI) \cdot 8%
}

\newcommand{\powerConsumtionDevelopmentFormulaApplied}{%
\begin{center}
\(\powerConsumtionDevelopmentFormula{3} = (\powerConsumtionTaskDefinitionDesign + \powerConsumtionTaskDefinitionDevelopment + \powerConsumtionTaskDefinitionTesting + \powerConsumtionCodingDesign + \powerConsumtionCodingDevelopment + \powerConsumtionCodingTesting) \cdot 8 = \totalPowerConsumtionDevelopment \, \days\)
\end{center}
}

\newcommand{\powerConsumtionDevelopmentEquation}{%
\begin{equation}\label{powerConsumtionDevelopmentEquation}
\powerConsumtionDevelopmentFormula{n}
\end{equation}
\begin{explanation}
где & $ n $ & количество этапов разработки программы; \\
    & $ \powerConsumtionTaskDefinitionI $ & трудоёмкость постановки задачи на \(i\)-м этапе разработки программы, дней; \\
    & $ \powerConsumtionCodingI $ & трудоёмкость программирования задачи на \(i\)-м этапе разработки программы, дней.
\end{explanation}
}


% Расчет стоимости машино-часа работы ПК
%% Energy hour price
\MULTIPLY{\pcEnergyConsumptionValue}{\energyConsumptionCostValue}{\electricityCostValue}

\newcommand{\energyHourCostFormula}{
\electricityCost = \pcEnergyConsumption \cdot \energyConsumptionCost
}

\newcommand{\energyHourCostEquation}{
\begin{equation}\label{energyHourCostEquation}
\energyHourCostFormula,
\end{equation}
\begin{explanation}
где & $ \pcEnergyConsumption $ & среднечасовое потребление электроэнергии \pc, \kvtShort; \\
    & $ \energyConsumptionCost $ & стоимость киловатт-часа электроэнергии, \ye
\end{explanation}
}

\newcommand{\powerConsumtionDevelopmentFormulaApplied}{%
\begin{center}
\(\electricityCost = \pcEnergyConsumptionValue \cdot \energyConsumptionCostValue = \electricityCostValue \, \ye\)
\end{center}
}



\newcommand{\pcHourCostFormula}{%
\cpuTimeCost = \electricityCost + \frac{\pcAmortizationCost + \pcRepairCost + \devPlaceAmortizationCost + \devPlaceRepairCost + \devPlaceRentCost}{\pcTimeFound}
}

\newcommand{\pcHourCostEquation}{
\begin{equation}\label{pcHourCostEquation}
\pcHourCostFormula,
\end{equation}
\begin{explanation}
где & $ \electricityCost $ & расходы на электроэнергию за час работы \pcye; \\
    & $ \pcAmortizationCost $ & годовая величина амортизационных отчислений на реновацию \pcye; \\
    & $ \pcRepairCost $ & годовые затраты на ремонт и техническое обслуживание \pcye; \\
    & $ \devPlaceAmortizationCost $ & годовая величина амортизационных отчислений на реновацию производственных площадей, \ye; \\
    & $ \devPlaceRepairCost $ & годовые затраты на ремонт и содержание производственных площадей, \ye; \\
    & $ \devPlaceRentCost $ & годовая величина арендных платежей за помещение, занимаемое \pcye; \\
    & $ \pcTimeFound $ & годовой фонд времени работы \pcye.
\end{explanation}
}