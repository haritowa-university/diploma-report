\FPeval{\totalCapitalExpensesValue}{clip(round((\capitalExpensesValue + \ppResultCostValue), 2))}

\newcommand{\totalCapitalExpensesFormula}{%
\totalCapitalExpenses = \capitalExpenses + \ppResultCost
}

\newcommand{\totalCapitalExpensesEquation}{
\begin{equation}\label{totalCapitalExpensesEquation}
\totalCapitalExpensesFormula,
\end{equation}
\begin{explanation}
где & $ \capitalExpenses $ & капитальные и~приравненные к ним затраты; \\
    & $ \ppResultCost $ & отпускная цена программы.
\end{explanation}
}

\newcommand{\totalCapitalExpensesFormulaApplied}{%
\begin{center}
\(\totalCapitalExpenses = \num{\capitalExpensesValue} + \num{\ppResultCostValue} = \num{\totalCapitalExpensesValue} \, \ye\)
\end{center}
}

