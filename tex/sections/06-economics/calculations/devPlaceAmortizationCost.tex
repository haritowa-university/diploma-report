\FPeval{\devPlaceBalanceCostResult}{\devPlaceMeterCostValue * \additionalDevPlaceCoefValue * \pcConsumedPlaceValue}
\FPeval{\devPlaceAmortizationCostResult}{clip(\devPlaceBalanceCostResult * \amortizationDevPlaceNormValue * 0.01)}

\newcommand{\devPlaceAmortizationCostFormula}{
\devPlaceAmortizationCost = \devPlaceMeterCost \cdot \additionalDevPlaceCoef \cdot \pcConsumedPlace \cdot \frac{\amortizationDevPlaceNorm}{100} = \devPlaceBalancePrice \cdot \frac{\amortizationDevPlaceNorm}{100}
}

\newcommand{\devPlaceAmortizationCostEquation}{
\begin{equation}\label{devPlaceAmortizationCostEquation}
\devPlaceAmortizationCostFormula,
\end{equation}
\begin{explanation}
где & $ \devPlaceMeterCost $ & цена квадратного метра производственной площади, \ye; \\
    & $ \additionalDevPlaceCoef $ & коэффициент, учитывающий дополнительную площадь; \\
    & $ \pcConsumedPlace $ & площадь, занимаемая ПК, \meterSquare; \\
    & $ \amortizationDevPlaceNorm $ & норма амортизационных отчислений на~производственные площади, \%; \\
    & $ \devPlaceBalancePrice $ & балансовая стоимость площадей, \ye \\
\end{explanation}
}

\newcommand{\devPlaceAmortizationCostFormulaApplied}{%
\begin{center}
\(\devPlaceAmortizationCost = \num{\devPlaceMeterCostValue} \cdot \num{\additionalDevPlaceCoefValue} \cdot \num{\pcConsumedPlaceValue} \cdot \frac{\num{\amortizationDevPlaceNormValue}}{\num{100}} = \num{\devPlaceAmortizationCostResult} \, \ye\)
\end{center}
}