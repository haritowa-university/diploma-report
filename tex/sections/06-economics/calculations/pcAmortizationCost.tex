%% Amortization PC renovation cost
\FPeval{\pcBalanceCostResult}{\pcCostValue * \priceIncreaseCoefValue * \mountCoefValue }
\FPeval{\pcAmortizationCostResult}{clip(\pcBalanceCostResult * \amortizationNormValue * 0.01)}

\newcommand{\pcAmortizationCostFormula}{
\pcAmortizationCost = \pcCost \cdot \priceIncreaseCoef \cdot \mountCoef \cdot \frac{\amortizationNorm}{100} = \pcCostBalance \cdot \frac{\amortizationNorm}{100}
}

\newcommand{\pcAmortizationCostEquation}{
\begin{equation}\label{pcAmortizationCostEquation}
\pcAmortizationCostFormula,
\end{equation}
\begin{explanation}
где & $ \pcCost $ & цена \pc на момент его выпуска, \ye; \\
    & $ \priceIncreaseCoef $ & коэффициент удорожания \pc; \\
    & $ \mountCoef $ & коэффициент, учитывающий затраты на монтаж и транспортировку \pc; \\
    & $ \pcCostBalance $ & балансовая стоимость \pcye; \\
    & $ \amortizationNorm $ & норма амортизационных отчислений на \pc.
\end{explanation}
}

\newcommand{\pcAmortizationCostFormulaApplied}{%
\begin{center}
\(\pcAmortizationCost = \pcCostValue \cdot \priceIncreaseCoefValue \cdot \mountCoefValue \cdot \frac{\amortizationNormValue}{100} = \pcAmortizationCostResult \, \ye\)
\end{center}
}