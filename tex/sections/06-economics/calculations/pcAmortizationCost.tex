%% Amortization PC renovation cost
\FPeval{\pcPcBalanceCostResult}{\pcPcCostValue * \priceIncreaseCoefValue * \mountCoefValue }
\FPeval{\pcAmortizationCostResult}{clip(\pcPcBalanceCostResult * \amortizationNormValue * 0.01)}

\newcommand{\pcAmortizationCostFormula}{
\pcAmortizationCost = \pcPcCost \cdot \priceIncreaseCoef \cdot \mountCoef \cdot \frac{\amortizationNorm}{100} = \pcPcCostBalance \cdot \frac{\amortizationNorm}{100}
}

\newcommand{\pcAmortizationCostEquation}{
\begin{equation}\label{pcAmortizationCostEquation}
\pcAmortizationCostFormula,
\end{equation}
\begin{explanation}
где & $ \pcPcCost $ & цена ПК на момент его выпуска, \ye; \\
    & $ \priceIncreaseCoef $ & коэффициент удорожания ПК; \\
    & $ \mountCoef $ & коэффициент, учитывающий затраты на монтаж и транспортировку ПК; \\
    & $ \pcPcCostBalance $ & балансовая стоимость \pcye; \\
    & $ \amortizationNorm $ & норма амортизационных отчислений на ПК.
\end{explanation}
}

\newcommand{\pcAmortizationCostFormulaApplied}{%
\begin{center}
\(\pcAmortizationCost = \num{\pcPcCostValue} \cdot \num{\priceIncreaseCoefValue} \cdot \num{\mountCoefValue} \cdot \frac{\num{\amortizationNormValue}}{\num{100}} = \num{\pcAmortizationCostResult} \, \ye\)
\end{center}
}