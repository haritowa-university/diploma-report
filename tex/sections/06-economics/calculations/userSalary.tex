
% \newcommand{\userRateFormulaApplied}{}

\FPeval{\userRateValue}{clip(round(((\userAverageSalaryFirstDegreeValue * \degreeSalaryCoefValue) / 170), 2))}

\newcommand{\userRateFormulaApplied}{%
\begin{center}
\(\userRate = \frac{\num{\userAverageSalaryFirstDegreeValue} \cdot \num{\degreeSalaryCoefValue}}{\num{170}} = \num{\userRateValue} \, \ye\)
\end{center}
}



\FPeval{\userSalaryValue}{clip(round((\pcTaskAccomplishTimeCostValue * \manualExecutionRateValue * \userRateValue * (1 + \bonuceRateCoefValue) * (1 + \additionalSalaryCoefValue) * (1 + \salaryFundCoefValue)), 1))}

\newcommand{\userSalaryFormula}{%
\userSalary = \pcTaskAccomplishTimeCost \cdot \manualExecutionRate \cdot \userRate \cdot (1 + \bonuceRateCoef) \cdot (1 + \additionalSalaryCoef) \cdot (1 + \salaryFundCoef)
}

\newcommand{\userSalaryEquation}{
\begin{equation}\label{userSalaryEquation}
\userSalaryFormula,
\end{equation}
\begin{explanation}
где & $ \pcTaskAccomplishTimeCost $ & время решения задачи на ПК, ч; \\
    & $ \userRate $ & среднечасовая ставка пользователя программы, \ye
\end{explanation}
}

\newcommand{\userSalaryFormulaApplied}{%
\begin{center}
\(\userSalary = \num{\pcTaskAccomplishTimeCostValue} \cdot \num{\manualExecutionRateValue} \cdot \num{\userRateValue} \cdot (1 + \num{\bonuceRateCoefValue}) \cdot (1 + \num{\additionalSalaryCoefValue}) \cdot (1 + \num{\salaryFundCoefValue}) = \num{\userSalaryValue} \, \ye\)
\end{center}
}