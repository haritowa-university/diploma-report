\FPeval{\userPCRentCostValue}{clip(round((\pcTaskAccomplishTimeCostValue * \manualExecutionRateValue * \cpuTimeCostResult), 2))}

\newcommand{\userPCRentCostFormula}{%
\userPCRentCost = \pcTaskAccomplishTimeCost \cdot \manualExecutionRate \cdot \cpuTimeCost
}

\newcommand{\userPCRentCostEquation}{
\begin{equation}\label{userPCRentCostEquation}
\userPCRentCostFormula,
\end{equation}
\begin{explanation}
где & $ \cpuTimeCost $ & стоимость одного машино-часа работы ПК.
\end{explanation}
}

\newcommand{\userPCRentCostFormulaApplied}{%
\begin{center}
\(\userPCRentCost = \num{\pcTaskAccomplishTimeCostValue} \cdot \num{\manualExecutionRateValue} \cdot \num{\cpuTimeCostResult} = \num{\userPCRentCostValue} \, \ye\)
\end{center}
}