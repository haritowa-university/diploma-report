%% PC hour cost
\FPeval{\cpuTimeCostResult}{\electricityCostResult + (\pcAmortizationCostResult + \pcRepairCostResult + \devPlaceAmortizationCostResult + \devPlaceRepairCostResult + \devPlaceRentCostResult) / (\pcTimeFundResult)}
\FPeval{\cpuTimeCostResult}{clip(round(\cpuTimeCostResult, 3))}

\newcommand{\pcHourCostFormula}{%
\cpuTimeCost = \electricityCost + \frac{\pcAmortizationCost + \pcRepairCost + \devPlaceAmortizationCost + \devPlaceRepairCost + \devPlaceRentCost}{\pcTimeFound}
}

\newcommand{\pcHourCostEquation}{
\begin{equation}\label{pcHourCostEquation}
\pcHourCostFormula,
\end{equation}
\begin{explanation}
где & $ \electricityCost $ & расходы на электроэнергию за час работы \pcye; \\
    & $ \pcAmortizationCost $ & годовая величина амортизационных отчислений на реновацию \pcye; \\
    & $ \pcRepairCost $ & годовые затраты на ремонт и техническое обслуживание \pcye; \\
    & $ \devPlaceAmortizationCost $ & годовая величина амортизационных отчислений на реновацию производственных площадей, \ye; \\
    & $ \devPlaceRepairCost $ & годовые затраты на ремонт и содержание производственных площадей, \ye; \\
    & $ \devPlaceRentCost $ & годовая величина арендных платежей за помещение, занимаемое \pcye; \\
    & $ \pcTimeFound $ & годовой фонд времени работы \pcye.
\end{explanation}
}

\newcommand{\pcHourCostFormulaApplied}{%
\begin{center}
\(\cpuTimeCost = \num{\electricityCostResult} + \frac{\num{\pcAmortizationCostResult} + \num{\pcRepairCostResult} + \num{\devPlaceAmortizationCostResult} + \num{\devPlaceRepairCostResult} + \num{\devPlaceRentCostResult}}{\num{\pcTimeFundResult}} = \num{\cpuTimeCostResult} \, \ye\)
\end{center}
}