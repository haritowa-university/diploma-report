\FPeval{\developerHourPriceValue}{clip(round((\productComplexityValue * \developerHourRateValue * (1 + \bonuceRateCoefValue) * (1 + \additionalSalaryCoefValue) * (1 + \salaryFundCoefValue)), 2))}

\newcommand{\developerHourPriceFormula}{%
\developerHourPrice = \productComplexity \cdot \developerHourRate \cdot (\num{1} + \bonuceRateCoef) \cdot (\num{1} + \additionalSalaryCoef) \cdot (\num{1} + \salaryFundCoef)
}

\newcommand{\developerHourPriceEquation}{
\begin{equation}\label{developerHourPriceEquation}
\developerHourPriceFormula,
\end{equation}
\begin{explanation}
где & $ \productComplexity $ & трудоёмкость разработки программного продукта, чел-ч; \\
    & $ \developerHourRate $ & среднечасовая ставка работника, осуществляющего разработку программного продукта, \ye; \\
    & $ \bonuceRateCoef $ & коэффициент, учитывающий процент премий в организации-разработчике; \\
    & $ \additionalSalaryCoef $ & коэффициент, учитывающий дополнительную заработную плату; \\
    & $ \salaryFundCoef $ & коёццифиент, учитывающий отчисления от фонда заработной платы.
\end{explanation}
}

\newcommand{\developerHourPriceFormulaApplied}{%
\begin{center}
\(\developerHourPrice = \num{\productComplexityValue} \cdot \num{\developerHourRateValue} \cdot (\num{1} + \num{\bonuceRateCoefValue}) \cdot (\num{1} + \num{\additionalSalaryCoefValue}) \cdot (\num{1} + \num{\salaryFundCoefValue}) = \num{\developerHourPriceValue} \, \ye\)
\end{center}
}