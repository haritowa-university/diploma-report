\FPeval{\annualExpensesValue}{clip(round((\userSalaryValue + \userPCRentCostValue), 2))}

\newcommand{\annualExpensesFormula}{%
\annualExpenses = \userSalary + \userPCRentCost
}

\newcommand{\annualExpensesEquation}{
\begin{equation}\label{annualExpensesEquation}
\annualExpensesFormula,
\end{equation}
\begin{explanation}
где & $ \userSalary $ & затраты на заработную плату пользователя программы; \\
    & $ \userPCRentCost $ & затраты на оплату аренды ПК при решении задачи.
\end{explanation}
}

\newcommand{\annualExpensesFormulaApplied}{%
\begin{center}
\(\annualExpenses = \num{\userSalaryValue} + \num{\userPCRentCostValue} = \num{\annualExpensesValue} \, \ye\)
\end{center}
}
