\subsection{Расчет себестоимости создания программного продукта}
\label{sec:economics:pppurecost}

Для определения себестоимости создания программного продукта необходимо определить затраты на заработную плату разработчика по формуле:
\developerHourPriceEquation

Среднечасовая ставка работника определяется исходя из единой тарифной системы оплаты труда в Республике Беларусь по следующей формуле:
\developerHourRateEquation

Определим среднечасовую ставку работника, подставив значения в формулу \formularef{developerHourRateEquation}:
\developerHourRateFormulaApplied

Также рассчитаем затраты на заработную плату разработчика по формуле \formularef{developerHourPriceEquation}:
\developerHourPriceFormulaApplied

Затраты на отладку программы определяются по формуле:
\softwareDebugCostEquation

Рассчитаем затраты на отладку программы, подставив значения в формулу \formularef{softwareDebugCostEquation}:
\softwareDebugCostFormulaApplied

Себестоимость разработки \gls{pp} определяется по формуле:
\ppCostEquation

Рассчитаем себестоимость разработки \gls{pp}, подставив вычисленные значения в формулу \formularef{ppCostEquation}:
\ppCostFormulaApplied

Таким образом, себестоимость разработки \gls{pp} равна \(\num{\ppCostValue} \, \ye\)