\subsection{Расчет стоимости машино-часа работы ПК}
\label{sec:economics:cpuclocktime}

Стоимтость машино-часа работы ПК определяется по формуле:
\pcHourCostEquation

Расходы на электроэнергию за час работы \pc, в свою очередь, определяются по формуле:
\energyHourCostEquation

Рассчитаем расходы на электроэнергию за час работы \pc, подставив значения в формулу \formularef{energyHourCostEquation}:
\energyHourCostFormulaApplied

Годовая величина амортизационных отчислений на реновацию \pc определяется по формуле:
\pcAmortizationCostEquation

Рассчитаем величину годовых амортизационных отчислений на реновацию \pc, подставив значения в формулу \formularef{pcAmortizationCostEquation}:
\pcAmortizationCostFormulaApplied

Годовые затраты на ремонт и техническое обслуживание ПК укрупнённо могут быть определены по формуле:
\pcSupportCostEquation

Рассчитаем годовые затраты на ремонт и техническое обслуживание \pc, подставив значения в формулу \formularef{pcSupportCostEquation}:
\pcSupportCostFormulaApplied

Годовая величина амортизационных отчислений на реновацию производственных площадей, занятых ПК, определяется по формуле:
\devPlaceAmortizationCostEquation

Используя формулу \formularef{devPlaceAmortizationCostEquation}, рассчитаем годовую величину амортизационных отчислений на реновацию производственных площадей, занятых ПК:
\devPlaceAmortizationCostFormulaApplied

Годовые затраты на ремонт и содержание производственных площадей укрупннённо могут быть определены по формуле:
\devPlaceSupportCostEquation

Рассчитаем годовые затраты на ремонт и содержание производственных площадей, подставив значения в формулу \formularef{devPlaceSupportCostEquation}:
\devPlaceSupportCostFormulaApplied

Годовая величина арендных платежей за помещение, занимаемое ПК, рассчитывается по формуле:
\devPlaceRentCostEquation

\(\devPlaceCommfrotCoef = \devPlaceCommfrotCoefValue\), так как помещение входит в разряд тех, которые расположены в цокольном этаже.

\(\devPlaceGeoEncreaseCoef = \devPlaceGeoEncreaseCoefValue\), так как помещение находится в городе Минске.

Рассчитаем годовую величину арендных платежей за помещение, занимаемое ПК, подставив значения в формулу \formularef{devPlaceRentCostEquation}:
\devPlaceRentCostFormulaApplied

Годовой фонд времени работы ПК определяется исходя из режима его работы по формуле:
\pcWorkabilityFundEquation

Рассчитаем годовой фонд времени работы ПК. подставив значения в формулу \formularef{pcWorkabilityFundEquation}:
\pcWorkabilityFundFormulaApplied

Рассчитаем стоимость машино-часа работы ПК, подставив вычисленные значения в формулу \formularef{pcHourCostEquation}:
\pcHourCostFormulaApplied