\subsection{Расчет стоимости машино-часа работы ПК}
\label{sec:economics:cpuclocktime}

Стоимтость машино-часа работы ПК определяется по формуле:
\pcHourCostEquation

Расходы на электроэнергию за час работы \pc, в свою очередь, определяются по формуле:
\energyHourCostEquation

Рассчитаем расходы на электроэнергию за час работы \pc, подставив значения в формулу \formularef{energyHourCostEquation}:
\energyHourCostFormulaApplied

Годовая величина амортизационных отчислений на реновацию \pc определяется по формуле:
\pcAmortizationCostEquation

Рассчитаем величину годовых амортизационных отчислений на реновацию \pc, подставив значения в формулу \formularef{pcAmortizationCostEquation}:
\pcAmortizationCostFormulaApplied

Годовые затраты на ремонт и техническое обслуживание \pc укрупнённо могут быть определены по формуле:
\pcSupportCostEquation

Рассчитаем годовые затраты на ремонт и техническое обслуживание \pc, подставив значения в формулу \formularef{pcSupportCostEquation}:
\pcSupportCostFormulaApplied