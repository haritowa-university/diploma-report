\subsection{Расчет годовых текущих затрат, связанных с эксплуатацией задачи}
\label{sec:economics:manualTotalExpenses}

Для расчёта годовых текущих затрат, связанных с эксплуатацией ПП, необходимо определить время решения данной задачи на ПК.

Время решения задачи на ПК определяется по формуле:
\pcTaskAccomplishTimeCostEquation

Время ввода в ПК исходных данных определяется по формуле:
\manualInputTimeEquation

Рассчитаем время ввода в ПК исходных данных, подставив значения в формулу \formularef{manualInputTimeEquation}:
\manualInputTimeFormulaApplied

Рассчитаем время решения задачи на ПК, подставив значения в формулу \formularef{pcTaskAccomplishTimeCostEquation}:
\pcTaskAccomplishTimeCostFormulaApplied

На основе рассчитанного времени решения задачи может быть определена заработная плата пользователя данного ПП. Затраты на заработную плату пользователя ПП определяются по формуле:
\userSalaryEquation

Рассчитаем среднечасовую ставку пользователя программы, подставив значения в формулу \formularef{developerHourRateEquation}:
\userRateFormulaApplied

Рассчитаем затраты на заработную плату пользователя ПП, подставив значения в формулу \formularef{userSalaryEquation}:
\userSalaryFormulaApplied

В состав затрат, связанных с решением задачи, включаются также затраты, свящанные с эксплуатацией ПК.

Затраты на оплату аренды ПК для решения задачи определяются по следующей формуле:
\userPCRentCostEquation

Рассчитаем затраты на оплату аренды ПК для решения задачи, подставив значения в формулу \formularef{userPCRentCostEquation}:
\userPCRentCostFormulaApplied

Годовые текущие затраты, связанные с эксплуатацией задачи, определяются по формуле:
\annualExpensesEquation

Рассчитаем годовые текущие затраты, связанные с эксплуатацией задачи, подставив значения в формулу \formularef{annualExpensesEquation}
\annualExpensesFormulaApplied