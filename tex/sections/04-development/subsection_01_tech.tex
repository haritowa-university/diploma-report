\subsection{Разработка архитектуры iOS-клиента}
\label{sec:development:arch:ios}

Перед описанием финальной архитектуры iOS-клиента стоит рассмотреть конкретные технологии, которые будут использоваться на~проекте.

\input{sections/04-development/paragraphs/swift}
\input{sections/04-development/paragraphs/rxswift}
\input{sections/04-development/paragraphs/realm}
\input{sections/04-development/paragraphs/rxfeedback}
\subsubsection{}
\label{sec:analysis:research:mobArch:ufeature}

По своей сути программные средства состоят из~набора функциональных возможностей. Обычно все возможности ПС реализованы в~пределах одного модуля. Результатом такого подхода является тесная связанность разных модулей, введение неявных зависимостей, увеличение сложности поддержки и~разработки новых функций. Паттерн \gls{mvvm} предназначен для~организации пользовательского интерфейса программного средства, однако часто код ПС можно условно разбить на~несколько независимых доменов -- обычно каждый такой домен называется микросервисом.

Микросервисы предоставляют минимальный, но необходимый для~интеграции \gls{api}, скрывая детали реализации. Для~примера в~iOS клиенте чата можно выделить следующие домены:

\begin{itemize}
	\item сервисы для~шифрования сообщений, общения с~сервером, работой с~базой;
	\item экраны списка контактов, настроек, списка диалогов и~конкретного диалога.
\end{itemize}

В 2016 году компания SoundCloud представила своё видение организации микросервисной архитектуры в~мобильных клиентах: \glspl{ufeature}.

\glspl{ufeature} -- архитектурный подход для~стуктурирования iOS клиентов, предоставляющий масштабируемость, оптимизацию времени сборки проетка и~циклов тестирования, гарантирующий адоптацию хороших практик разработки в~команде. Основной идеей подхода является разработка независимых возможностей клиента, которые взаимодействуют при помощи чётко обозначенного \gls{api}\cite{soundcloud:ufeature}.

Типичная \gls{ufeature} представляет из~себя отдельный проект с~4 схемами:

\begin{itemize}
	\item код и~ресурсы;
	\item тесты;
	\item данные, которые используются в~тестах и~примере;
	\item пример использования, оформленный в~виде программного средства.
\end{itemize}

На рисунке \ref{sec:analysis:research:mobArch:ufeature:featureDependencyDiagram} представлена связь схем одного модуля \gls{ufeature}.

\begin{figure}[h]
  \centering
    \includegraphics{inc/img/ufeature-diagram.png}
  \caption{Организация схем внутри модуля одной uFeature}
  \label{sec:analysis:research:mobArch:ufeature:featureDependencyDiagram}
\end{figure}

Автор подхода предлагает разделять все \gls{ufeature} на~два вида:

\begin{itemize}
	\item \emph{Foundation} -- основные строительные блоки, общие расширение пользовательского интерфейса, сервисы;
	\item \emph{Product} -- возможности, с~которыми пользователь взаимодействует и~видит на~экране, например список диалогов;
\end{itemize}


\input{sections/04-development/paragraphs/modules}

В пунктах \ref{sec:analysis:research:mobArch:mvc} и~\ref{sec:analysis:research:mobArch:mvvm} были рассмотренны основные решения для~построения архитектуры пользовательского интерфейса в~iOS приложениях, а~в~пункте \ref{sec:analysis:research:mobArch:ufeature} -- способ организации модульного приолжения. За основну архитектуры будущего приложения было принято выбрать модифицированный паттерн \gls{mvvm}, код приложения будет организован в~модули, разбитые на~несколько git репозиториев. 