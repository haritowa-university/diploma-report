\subsubsection{}
\label{sec:development:client:db}

Модуль базы данных содержит в себе набор моделей и методов, выполняющих запросы в базу данных или как-то её изменяющих. 
Как уже отмечалось в пункте \ref{sec:development:arch:ios:realm}, для описания таблицы в мобильной базе данных \textit{Realm}, достаточно создать новый тип на языке Swift и унаследовать его от типа \textit{Object}. Пример модели базы представлен в листинге \ref{sec:development:client:db:code:model}, а пример запроса в листинге \ref{sec:development:client:db:code:query}.

\begin{code}
	\inputminted{swift}{inc/src/db_model.swift}
   \caption{Модель устройства в базе данных}
   \label{sec:development:client:db:code:model}
\end{code}

\begin{code}
	\inputminted{swift}{inc/src/db_query.swift}
   \caption{Запрос списка устройств конкретного пользователя}
   \label{sec:development:client:db:code:query}
\end{code}

Помимо списка моделей и запросов, слой базы данных содержит вспомогательный класс \textit{DB}, который позволяет получить доступ к конкретным сущностям \textit{Realm} для текущего или главного потока исполнения, а так же обобщённый метод для записи моделей в базу, который продемонстрирован в листинге \ref{sec:development:client:db:code:util}.

\begin{code}
	\inputminted{swift}{inc/src/db_util.swift}
   \caption{Метод для реактивной записи моделей в базу}
   \label{sec:development:client:db:code:util}
\end{code}