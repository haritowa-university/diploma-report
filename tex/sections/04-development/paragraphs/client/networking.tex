\subsubsection{}
\label{sec:development:client:networking}

Сетевой слой состоит из набора моделей запросов и ответов, описания \gls{http} \gls{api}, методы для отправки запросов, \gls{ws} клиент.

Для парсинга JSON используется библиотека \textit{ObjectMapper}, которая позволяет при помощи операторов выразительным образом описать процесс сериализации и десереализации модели во многие форматы.
В листинге \ref{sec:development:client:networking:code:deserialize} продемонстрировано описание десериализации модели пользователя и списка пользователей, а в листинге \ref{sec:development:client:networking:code:serialize} -- сериализация события отпраки сообщения.

\begin{code}
	\lstinputlisting{inc/src/deserialization.swift}
   \caption{Десериализация модели пользователя и списка контактов}
   \label{sec:development:client:networking:code:deserialize}
\end{code}

\begin{code}
	\lstinputlisting{inc/src/serialization.swift}
   \caption{Сериализация модели события отпраки сообщения}
   \label{sec:development:client:networking:code:serialize}
\end{code}

Как уже отмечалось в пункте \ref{sec:development:arch:pp:communication}, всё общение клиента с сервером происходит по определённому формату, для упрощения процесса сериализации, этот формат инкапсулирован в обобщённый тип \texttt{ResponseContainer}, который продемонстрирован в листинге \ref{sec:development:client:networking:code:deserialize:container}.

\begin{code}
	\lstinputlisting{inc/src/parsing_container.swift}
   \caption{Контейнер для десериализации данных}
   \label{sec:development:client:networking:code:deserialize:container}
\end{code}

Тип \textit{ChattySocket} отвечает за взаимодействие с сервером при помощи \gls{ws}. В качестве имплементации общения по протоколу \gls{ws} используется библиотека \textit{Starscream}, тип \textit{ChattySocket} является реактивной обёрткой, который добавляет поддержку авторизации, отправки запросов с использованием моделей приложения, аггрегацию ошибок и мониторинг состояния сети.

Для описания \gls{http} \gls{api} приложения используется библиотека \textit{Moya} и конкретное описание \gls{api}: \textit{ChattyAPI}. Основными задачами типа являются:
\begin{enumerate}
	\item Описание всех эндпойнтов приложения.
	\begin{enumerate}
		\item Описание пути запроса.
		\item Описание типа запроса (\textit{GET}, \textit{POST} и так далее).
		\item Ответы-заглушки на каждый запрос.
		\item Способ сериализации параметров запроса.
		\item Список параметров запроса.
		\item Способ отправки запроса, необходимый для работы \textit{NSUrlSession}.
	\end{enumerate}
	\item Базовый путь к \gls{api} приложения.
	\item Добавление ко всем запросам необходимых заголовков, в том числе авторизации.
	\item Общая обработка ошибок сети, сервера и запросов.
\end{enumerate}

Благодаря предоставлению ответов-заглушек, стало возможным использование \textit{MockMoyaProvider}, который используется в тестах для иммитации отправки запросов и получения ответов.