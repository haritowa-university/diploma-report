\subsubsection{}
\label{sec:development:client:vc}

Большую часть кода приложения занимает пользовательский интерфейс. Как отмечалось в пункте \ref{sec:analysis:research:mobArch:mvvm}, за основу пользовательского интерфейса в приложении выбран паттерн \gls{mvvm}. 
На рисунке \ref{sec:development:client:vc:code:vm} приведена асбтрактная модель представления, а на рисунке \ref{sec:development:client:vc:code:vc} -- базовый контроллер, который поддерживает внедрение модели прдставления. Большая часть пользовательского интерфейса описывается при помощи \textit{Interface Builder}, который сериализует результат работы разработчика в XML файлы. Для поддержки внедрения модели представления, контроллеру необходимо реализовать два метода: 
\begin{enumerate}
	\item \textit{generateInput}, в котором контроллер генерирует необзодимы набор внешних биндингов, который использует модель представления.
	\item \textit{setup}, в котором контроллер производит установку биндингов свойств модели к свойствами представления.
\end{enumerate}

\begin{figure}[h]
	\lstinputlisting{inc/src/vm.swift}
   \caption{Абстрактная модель представления}
   \label{sec:development:client:vc:code:vm}
\end{figure}

\begin{figure}[h]
	\lstinputlisting{inc/src/vc.swift}
   \caption{Абстрактный тип, поддерживающий внедрение модели представления}
   \label{sec:development:client:vc:code:vc}
\end{figure}

Механизм внедрения модели представления поддерживает отложенный вызов, что необходимо при работе с представлениями фреймворка \gls{uikit}, так как контроллеры и объекты представления являются невалидными до определённого этапа жизненного цикла.