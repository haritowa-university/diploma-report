\subsubsection{}
\label{sec:development:client:encryption}

Первым разработанным в приложении модулем является модуль криптографии, так как для его работы не нужны никакие дополнительные специфические для приложения зависимости.

Основным классом, предоставляющим криптографическое \gls{api} является \texttt{Crypt}. Класс предоставляет широкий набор возможностей:

\begin{itemize}
	\item \textit{RSA} шифрование и дешифрование;
	\item \textit{AES} шифрование и дешифрование;
	\item генерация \textit{AES} и \textit{RSA} ключей;
	\item деривацию \textit{AES} ключа из строки (используется для генерации ключа из пароля приложения);
	\item сериализацию и десереализацию публичного \textit{RSA} ключа в \textit{PEM} формат;
	\item абстрагирует работу с iOS технологией для безопасного хранения значений \textit{Keychain};
	\item набор констант, обильно использующихся во всех криптографических процессах приложения;
	\item вспомогательные методы для сохранения и чтения криптографической личности пользователя.
\end{itemize}

Также класс вводит несколько алиасов типов:
\begin{itemize}
	\item \texttt{AESKey} -- массив байт, представляющих собой ключ для \textit{AES} шифрования;
	\item \texttt{RSAPrivateKey} -- массив байт, представляющих собой приватный ключ для \textit{RSA} шифрования;
	\item \texttt{RSAPublicKey} -- массив байт, представляющих собой публичный ключ для \textit{RSA} шифрования.
\end{itemize}

Для упрощения разработки приложения, используется библиотека \textit{SwCrypt}.

Также модуль криптографии содержит в себе набор возможных ошибок, которые происходят при работе с \gls{api} модуля, который отражён в листинге \ref{sec:development:client:encryption:code:errors}.

\begin{code}
	\inputminted{swift}{inc/src/crypto_errors.swift}
   \caption{Набор возможных ошибок при работе с API модуля криптографии}
   \label{sec:development:client:encryption:code:errors}
\end{code}

Последним типом в модуле криптографии является \texttt{UserCryptIdentity}, который отражает криптографическую личность пользователя:
\begin{itemize}
	\item публичный и приватный \textit{RSA} ключи;
	\item \textit{PEM} публичного ключа пользователя;
	\item \textit{AES} ключ и начальный вектор для локального содержимого.
\end{itemize}