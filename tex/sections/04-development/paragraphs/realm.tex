\subsubsection{}
\label{sec:development:arch:ios:realm}

\emph{Realm} --- система управления базой данных, нацеленная на использованияе в приложениях для мобильных устройств(Android, iOS, Xamarin, React Native). Дизайн базы использует парадигму реактивного программирования, многие классы являются <<живыми>>: обновляются вместе с изменением базы и способны высылать уведомления об изменившихся данных. К основным особенностям Realm можно отнести:

\begin{itemize}
	\item работа с базой происходит с использованием моделей, написанных разработчиком на привычном языке программирования;
	\item данные хранятся в бинарном формате, схемой для которого выступают классы-модели;
	\item база поддерживает связи один к одному, один ко многим, создание связи между типами автоматически создаёт обратную связь;
	\item поддержка индексируемых полей;
	\item поддержка первичных ключей;
	\item частичная поддержка NSPredicate на iOS;
	\item организация ленивого чтения из базы при помощи прокси-объектов, что с одной стороны позволяет использовать API так, будто чтение не является ленивым, а с другой -- обойтись без лишних вычислений;
	\item реактивный подход, который выражен в виде <<живых>> запросов, объектов и возможности получать уведомления об изменениях данных (на iOS существует интеграция с \gls{kvo});
	\item <<точные уведомления>> -- механизм Realm, позволяющий вместе с уведомлением об изменениях получить список изменений, адаптированный для обновления UI-коллекций;
	\item потокобезопасность.
\end{itemize}

Существует большое количество сторонних библиотек, использующих расширяющих Realm, на проекте используются:

\begin{itemize}
	\item \emph{RxRealm} -- обёртка над \gls{api} Realm, позволяющая превращать объекты и запросы в \gls{observable} и описывающая операции записи/удаления в виде \gls{observable};
	\item \emph{RxRealmDataSources} -- адаптер Realm для механизма биндинга данных к UI-коллекциям, реализованном в RxCocoa.
\end{itemize}