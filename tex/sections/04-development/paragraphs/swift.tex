\subsubsection{}
\label{sec:development:arch:ios:swift}

\emph{Swift} --- язык программирования общего назначения, разрабатываемый с современным подходом к безопасности, производительности и паттернам дизайна программного обеспечения.\cite{swift:about} Заявлена поддержка платформ компании Apple(iOS, watchOS, tvOS, macOS и любые будущие), Linux и многих других в тестовом режиме. Компиляция происходит при помощи \gls{llvm}, поэтому у языка большой потенциал для покорения новых платформ.

Целью проекта Swift является создание языка для использования в различных направлениях: от системного программирования, до мобильных приложений, масштабирующегося до объёмов облачных сервисов. Основные принципы, которыми руководствуются дизайнеры и разработчики языка:

\begin{enumerate}
	\item \emph{Безопасность}. Язык всячески старается избегать неопределённого поведения, снимает с разработчика необходимость беспокоиться о работе с указателями, напрямую с памятью, заставляет обрабатывать(или игнорировать, явно это отмечая) большинство мест, уязвимых для ошибок разработчиков.
	\item \emph{Скорость}. Язык предполагается как альтернатива языкам, основанным на C(C, C++, Objective-C), что автоматически поднимает вопрос его способности составить им конкуренцию в производительности. Swift держит высокую планку предсказуемой производительности(нет неожиданных понижений производительности на различные процессы языка и \gls{sdk}, например нет \gls{gc}).
	\item \emph{Выразительность}. Swift использует десятилетия достижений в Информатики, предоставляя синтаксис, который приносит удовольствие при разработке и чтении, поставляя современные возможности.  
\end{enumerate}

Язык обладает многими возможностями, которые упрощают написание и чтение кода, по прежнему предоставляя разработчику контроль, необходимый для системного программирования. Swift поддерживает вывод типов, модульность, управление памятью является автоматическим, неполный список особенных и отлично реализованных возможностей языка:

\begin{enumerate}
	\item замыкания, унифицированные для совместного использования с указателями на функции;
	\item множественные возвращаемые результаты из функций, поддержка Tuples на уровне языка;
	\item обобщения;
	\item гибкие структуры, поддерживающие методы, расширения и протоколы;
	\item паттерны мира функционального программирования;
	\item эффективная система обработок ошибок, интегрированная в систему типов;
	\item паттерн матчинг;
	\item продвинутые механизмы для контроля потока выполнения;
	\item неизменяемые переменные и типы данных;
	\item optional и набор синтаксического сахара для удобной работы с ним.
\end{enumerate}

Язык является компилируемым, имеет поддержку REPL, расширяет стандартный подход к использованию \gls{llvm}(компиляция производится в два этапа в SIL, LLVM IR, с набором оптимизаций на каждом).

В языке широко используется парадигма Value Type, которая разделяет все типы на ссылочные и типы значений. Типы значений передаются в методы по значению(копируются), ссылочные -- по ссылке. Копирование значения позволяет разработчику меньше беспокоиться о непредвиденных мутациях данных, одновременном доступе к данным с разных потоков.

Частью языка является оптимизация \gls{cow}, копирующая значения только при их непосредственном изменении. Оптимизация отлично сочетается с типами значений, позволяя сохранить производительность ссылочных типов. Механизм достаточно умён для определения одинаковых значений, что позволяет не выделять место для изменённых данных, если в памяти уже имеется блок данных с таким же значением. Язык позволяет встроить \gls{cow} в собственные типы данных. При выполнении кода в листинге \ref{sec:development:arch:ios:swift:code:cow} для переменных x и z не будет аллоцироваться память, а для y -- будет.

\begin{code}
	\inputminted{swift}{inc/src/swift_cow_example.swift}
   \caption{Пример кода, подлежащего COW оптимизации}
   \label{sec:development:arch:ios:swift:code:cow}
\end{code}

В языке поддерживаются два вида алгебраических типов данных:

\begin{itemize}
	\item \emph{Sum Types}(тип-сумма). Типы данных, все возможные значения которых вычисляются путём \textit{суммирования} количества всех возможных значений внутренних типов данных;
	\item \emph{Product Types}(тип-произведение). Типы данных, все возможные значения которых вычисляются путём получения мощности \textit{декартового произведения} всех возможных значений внутренних типов данных.
\end{itemize}

Примерами типов-произведения являются структуры и классы, типов-суммирования -- Tuple и перечисления. Язык имеет мощную систему перечислений, позволяющую установить взаимное соответствие между элементами перечисления и значениями любого типа, задать каждому элементу перечисления набор ассоциированных значений. Пример типа-суммы представлен в листинге \ref{sec:development:arch:ios:swift:code:barcode}.

\begin{code}
	\inputminted{swift}{inc/src/swift_enum_sample.swift}
   \caption{Пример типа-суммы в Swift}
   \label{sec:development:arch:ios:swift:code:barcode}
\end{code}