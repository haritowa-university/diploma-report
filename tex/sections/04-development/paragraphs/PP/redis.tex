\subsubsection{}
\label{sec:development:arch:pp:redis}

Поскольку \textit{Vapor} предоставляет гибкую \gls{orm}, вопрос выбора конкретной базы данных было решено оставить до момента нагрузочного тестирования приложения и тестирования корректности работоспособности: в любой момент можно легко заменить базу данных на любую другую, к которой написан соответствующий дравер для \textit{Vapor Fluent}.

Часто для повышения производительности приложения и общего упрощения кода, некоторые обязанности базы данных перекладываются на инструменты кеширования данных в памяти. Стандартом в индустрии является \textit{Redis}.

\textit{Redis} -- хранилище структур данных, использующее оперативную память, которое применяется в качестве базы данных, кеша и брокера сообщений. Хранилище поддерживает следующие типы данных:

\begin{itemize}
	\item строка;
	\item хэш;
	\item список;
	\item множество;
	\item упорядоченные множества;
	\item карты битов;
	\item и некоторые другие.
\end{itemize}

Имеется поддержка репликации, скриптования при помощи языка LUA, транзакционности и различные уровни сериализации данных на диск. На поддерживаемых типах можно атомарно выполнять множество операций, например:

\begin{itemize}
	\item конкатенация строк;
	\item увеличение значения хэша;
	\item добавление элемента в список;
	\item вычисление пересечения множеств;
	\item сумма и разность множеств;
	\item получение элемента с самым высоким рейтингом в упорядоченном множестве.
\end{itemize}

Инструмент показывает прекрасную производительность и конфигурируемость, поддерживает персистентность в различных режимах и имеет лучшие показатели надёжности в рамках решаемых задач.