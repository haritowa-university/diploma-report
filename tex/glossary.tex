\newacronym{rx}{Rx}{Reactive Extensions}
\newacronym{mvc}{MVC}{Model-View-Controller}
\newacronym{mvvm}{MVVM}{Model-View-ViewModel}
\newacronym{api}{API}{Application Programming Interface}
\newacronym{ide}{IDE}{Integrated development environment}

\newacronym{cow}{COW}{Copy-on-Write}
\newacronym{llvm}{LLVM}{Low Level Virtual Machine}
\newacronym{sdk}{SDK}{Software development kit}
\newacronym{gc}{GC}{Garbage Collector}
\newacronym{kvo}{KVO}{Key-Value Observing}
\newacronym{voip}{VOIP}{Voice over IP}
\newacronym{orm}{ORM}{Object-relational mapping}
\newacronym{di}{DI}{Dependency Injection}
\newacronym{http}{HTTP}{Hypertext Transfer Protocol}

\newglossaryentry{uikit}
{
  name={UIKit},
  description={Часть фреймворка Cocoa Touch, разрабатываемого Apple для платформы iOS и tvOS. Предоставляет необходимую инфраструктуру для разработки графического интерфейса.}
}

\newglossaryentry{pure-function}
{
  name={Чистая функция},
  description={Подпрограмма, которая не имеет побочных эффектов и отображает входные данные на результат, то есть для одинаковых аргументов всегда возвращает одинаковый результат}
}

\newglossaryentry{pattern}
{
  name={Паттерн},
  description={Повторяемая архитектурная конструкция, представляющая собой решение проблемы проектирования в рамках некоторого часто возникающего контекста}
}

\newglossaryentry{observable}
{
  name=Observable,
  description={Поток данных, способный оповестить своих подписчиков о новых значениях, ошибке или успешном окончании}
}

\newglossaryentry{observer}
{
  name=Observer,
  description={Потребитель потока данных или отдельных событий}
}

\newglossaryentry{ufeature}
{
  name=uFeature,
  plural={uFeatures},
  description={Архитектурный паттерн от компании SoundCloud, устанавливающий правила для организации монолитного приложения в виде набора слабо связанных сервисов}
}

\newglossaryentry{aes}
{
  name=AES,
  description={Симметричный алгоритм шифрования, использующийся в большинстве современных криптографических реализациях криптографических систем}
}

\newglossaryentry{ws}
{
  name=WebSocket,
  description={Протокол полнодуплексной связи (может передавать и принимать одновременно) поверх TCP-соединения, предназначенный для обмена сообщениями между клиентом и сервером в режиме реального времени}
}


\glsaddall

% \sectioncentered*{Определения и сокращения}
% \addcontentsline{toc}{section}{Определения и сокращения}

% \printglossaries